\section{Theory}
\label{polanyi}

\subsection{intro}

\subsection{Knowledge: Nature and Conveyance}

To deal with the conduit prior to dealing with the contents would be a senseless endeavour. Consequently, to accurately examine knowledge and skills transmission in the rather informal and ``laid back'' atmosphere fostered by hackerspaces, it is first necessary to explore the nature of knowledge itself: what it is, what it is perceived to be and how it has been historically regarded, particularly during and after the Western Enlightenment. In witnessing the effects of the popularisation of the Internet in terms of its (quite often realised) potential for de-centralisation, the rise of hackerspaces as real-life venues for discussion, learning and socialisation seems odd.

Indeed, the de-centralising power of the Internet and other media has been thoroughly studied in a variety of scenarios, the most common being distance learning and free/open source software production.

A frequently cited resource, \textit{The No Significant Difference Phenomenon}, by Thomas Russel \citeyearpar{russel99} brings to the reader's attention 355 studies, dissertations, papers and reports (conducted from as early as 1928) whose main argument is that distance education is as effective as what the author refers to as the ``traditional classroom''. The report's overwhelming conclusions should, however, be taken with a grain of salt, as it is obvious that not only does it seem written by an advocate rather than an independent researcher, but it also clearly lacks any counter-argumenting studies that would be central to a balanced discussion.

For a more dispassionate argument, one can mention look at the work of \citet{clark94}, who takes a rather agnostic approach to the issue of the role of new technologies in learning, arguing that media itself is not the issue, but, rather, method is: ``media research is a triumph of enthusiasm over substantive examination of structural processes in learning and instruction'' \citep[p.27]{clark94}. \citeauthor{clark94} criticises what he sees as baseless excitement, not because he deems teaching methods based in new media (and distance learning) inadequate, but simply because, he argues, the methodologies of most studies have focused on \textit{vehicle} as opposed to \textit{substance}.

At the other end of the spectrum are those authors who maintain that distance education, in particular through the use of new technologies such as the Internet, simply cannot match the face-to-face experience, either delivered inside the traditional classroom or through more empirical methods or apprenticeships. \citepos{farber98} critique of the Russel report, for instance, lies upon the argument that most of its studies focus exclusively on ``measurable competence'' while completely neglecting to consider other complementary experiences, brought about by presential and face-to-face contact typical of traditional methods, stating that, through such experiences, one can acquire:

\begin{quote}
\ldots some understanding and taste for what people like to call "the life of the mind" ---a deceptive term, perhaps, for something [\dots] felt to be so deeply rooted in physical community and place, and that, far from being bodiless, incorporated itself [\ldots] in images of faces, gestures, settings, voices; in qualities of light and kinds of weather.
\end{quote}

While the quotation might seem charged with an element of nostalgia, the author's argument seems sound: the learning experience in traditional environments extends beyond formal curricula. It includes experiences, relationships, networks. Yet, even instinctively, one can get an impression of the presence of an additional factor reinforcing this concept; something beyond what can be spoken, written or enacted. Something \emph{implicit} that makes physical contact a significant part of the learning experience.

That instinctive element, I hypothesise, is precisely the main cohesive factor bringing hackers together, back from the abstract confines of the Internet to the physical spaces where they now meet. But what exactly is it? In 1951, Chemist turned Philosopher and Social Scientist Michael Polanyi came up with the concept of \textit{Tacit Knowledge} to attempt to explain why some aspects of human knowledge are increasingly difficult (or simply impossible) to transfer solely by means of written and oral communication. In his time, Polanyi challenged the prevailing rationalistic science paradigm, inherited from the times of the Enlightenment, whereby \textit{that which was known} was considered independent from \textit{those who knew it}. In simple terms, the concept is usually illustrated with the example of bicycle-riding. No amount of written of spoken knowledge will, by itself, suffice as a feasible attempt to transfer the skills necessary to do it.

\subsection{One epistemology. Many implications}

The transmission of certain skills and knowledge, argued Michael Polanyi, requires a cooperative process of dialogue and unspoken understanding made possible by human contact through physical proximity, interaction and a certain degree of cultural affinity. Of course, this postulate is not new. Since Plato (and perhaps earlier), philosophers have identified a human dimension involved in learning. Polanyi's angle, however, is unique in a number of ways. Indeed, to fully apprehend it, one must examine these within a historical frame.

First, Polanyi dared to challenge a prevailing epistemological paradigm, (referred to as ``the objective ideal'') that became prevalent as a result of the ``scientific triumphs'' \citep[p.17]{gelwick77} that took place after the Copernican revolution but the causes of which began far earlier, with several historical developments that were, in his view, instrumental in leading to it.

Amongst them was the gradually increasing tension between two concepts he described as moral scepticism and moral perfectionism. From Greece onwards, the phenomenalist view that moral knowledge was unat\-tainable continued and grew in the works of modern thinkers and philosophers such as the Marquis de Sade, Nietzche and Rimbaud. Modern existentialism, Polanyi argued, uses moral scepticism to attack social morality as artificial and hypocritical. This progressive devaluation of moral values, however, could not by itself sway society on its way towards the mass adoption of the objective ideal: the degradation of judeo-christian morality caused partly by the advent of scientific rationalism led some to pursue such values even more passionately. Attacks on christianity during the Enlightenment led to the transposition of moralistic dogmas away from religion and ``into man's secular thoughts'' \citep[p.57]{polanyi66}, thus creating a strange collision between the rationality and objectivity sought by the nascent enlightened society with a secularised moral demand for social improvement. 

While Polanyi conceded that this tension initially led to the improvement of ``almost every human relationship, both private and public'', he also claimed that the increasing radicalisation of these views not only posed ``dangerous internal contradictions'' \citep[p.58]{polanyi66} but had led to the rise of existentialism, nihilism and further, Marxism, the tragic result being the discredit of ``all explicit expressions of morality''. The tension between the two sides, thus, resulted in a ``paradox of morality'' which ultimately led modern society to seek solace in absolute and almost thoughtless objectivity.

Second in the set of consequences leading to the wide adoption of the objective ideal (and the dawn of the ``modern scientific revolution'') was, according to Polanyi, the mechanisation of society. By means of a simple but eloquent account of the process leading to the growth of mechanism, he discussed this point with a brief historic recount, starting from Greek times, specifically Pythagoras, who came to view the world ``exclusively in terms of numbers''. Numerical relationships as beautiful as those found within triangles, thought Pythagoras, were present in all aspects of the world. Post Copernicus (and even more so after Kepler) the universe came to be seen as a mechanical system running with clock-work precision ---a fact that delighted those who studied it. \citet[p.7]{polanyi58} noted that Kepler regarded his own discoveries in ``ecstatic communion'', amazed and bewildered by the universal precision found in god's creation. Yet, after Galileo and Newton, things changed. Rather than being an inherent universal force, mathematics went on to be regarded simply as a summary of experience:

\begin{quote}
Numbers and geometrical forms are no longer assumed to be inherent as such in Nature. Theory no longer reveals perfection; it no longer contemplates the harmonies of Creation \ldots `pure' mathematics, formerly the key to nature's mysteries, became strictly separated from the \emph{application} of mathematics to the formulation of empirical laws. \citep[p.28]{polanyi58}
\end{quote}

What was to follow would be an almost inevitable progression from Descartes' rationalism to zealous objectivity marked the decline of the Enlightenment and the dawn of Mechanism, as proclaimed by many voices, the loudest of which was that of Ernst Mach, from the Vienna school of positivism and his book \textit{Die Mechanik}, published in 1883. 

It was to be Polanyi, then, who would restore confidence in human intuition by demonstrating an ever-present human element intrinsic in the act of knowing:

\begin{quote}
Both a craftsman and a novice are capable of identifying a particular tool and indicating some of its functions, but what distinguishes the craftsman's knowledge from that of the novice lies in the former's ability to use the tool subsidiarily in order to focus on the object of his craft. In the case of the scientist, he is not only identified by his explicit knowledge of given premises, but by his subsidiary utilization of those premises in the practice of science \citep[pp23-24]{kane84}.
\end{quote}

This analogy is particularly enlightening in the sense that it illustrates how Polanyi's thesis has extensive implications within today's prevailing systems for the acquisition of knowledge. Polanyi's unusual career pathway ---from the exact sciences to sociology and philosophy--- made him acutely aware of the need to ground his precepts on modern foundations, applicable to and compatible with modern practices and standard conventions of generally accepted scientific practices. Indeed, it was the scientific process itself that he sought to reform, for he reckoned that its very conception relied on principles which were ignored and sometimes expressly denied by many of his colleagues. 

Polanyi understood that no scientific theory can (nor should it) achieve perfect objectivity. As such, the boundaries of knowledge are ever broadened by reliance on certain presuppositions ---sets of which he dubbed ``subsidiary knowledge'', reached by using the senses as ``clues'': vehicles through which one can transcend into more abstract realms, ultimately leading to the acquisition of focal awareness or focal knowledge. Both types of awarenesses ---focal and subsidiary--- are necessary for new knowledge to spur.

Each new discovery (acquisition of explicit knowledge) is accompanied by a number of ``premises'' which do not (and cannot) stand on their own and that, until proven, remain as tacit or a-critical pieces of subsidiary awareness. Yet, they are essential to broadening the boundaries of scientific comprehension. Polanyi judged the almost obsessive scientific strive for objectivity to be misguided, arguing that all aspects of scientific discovery: data, theory and experimentation relied not only on said unproven subsidiary knowledge but also on the personal interpretation of these elements. Polanyi described these ``premises'' as by-products of proven theories ---by becoming proven themselves, they move into the realm of the explicit. If they are disproved, though, they are replaced, often generating Kuhn-style paradigm shifts.

In making such bold affirmations, however, Polanyi ran into an apparent dilemma, as, at a glance, it would sound like a recipe for the advocacy of blind faith and/or the lenient type of ``science'' practised by fanatics and supporters of pseudo-scientific belief systems. Indeed, insofar as personal interpretation (or subjectivity) is regarded as an integral part of the scientific method, all claims, including unsubstantiated or even fraudulent ones, would gain an aura of validity which would, very rapidly, over-flood the gates of serious inquiry to the point where it would ``run the risk of discontinuity and ultimately, inertia'' \citep[p.37]{kane84}. 

As a consummate and respected scientist, Polanyi was firmly committed to the rigorous experimental techniques of accepted scientific practices and intimately familiar with their minutiae, virtues and flaws. Consequently he was perfectly aware of this apparent predicament, which he addressed by means of the application of a concept he dubbed the ``conviviality of science'' whereby said gates are guarded by scientific communities, bound together by common thought, solid standards, deep tradition and a desire for accuracy and rectitude\footnote{One could replace these last two concepts with, simply, ``the truth''. However, keeping with Polanyi's canons an absolute concept such as truth does not do the point presented any favours.}. 

Ever the scientist, Polanyi sought to improve the act of knowing by correctly identifying its idiosyncrasies, a fact that marks the second particular aspect that differentiates his epistemological system from preceding ones also considering a human element, as it is grounded on present-day reasoning and meant to aid understanding and resolving the present-day problems associated with the acquisition of knowledge: ``man can transcend his own subjectivity by striving passionately to fulfil his personal obligations to universal standards'' \citep[p.4]{polanyi58}.

A third aspect differentiating Polanyi's theory from earlier ones is its advocacy for strong independence as a necessary condition for the production and transmission of knowledge. Polanyi recounts his conversations with a ``leading communist theoretician'' by the name of Bukharin\footnote{We assume that Polanyi refers to Nikolai Bukharin, the Soviet politician executed by Stalin in 1938.} \citep{polanyi66}, who argued that pure research ---\textit{scientia gratia scientis}--- was ``a morbid symptom of class society''. Polanyi was appalled by Bukharin's appraisal, which he saw as a reflection of what he considered to be an over-mechanised and utilitarian society, where express and useful purpose was required for all actions, including the pursuit of knowledge. Relegating research activities to central control, he argued, significantly decreased their effectiveness, as a certain degree of flexibility vastly improves the success of epistemological ventures. Communism had no room for the type of serendipitous discoveries that lead Pasteur to give birth to the field of immunology and Flemming to discover Penicillin.

Polanyi viewed state-controlled science as an early warning of the advent and growth of totalitarianism across Europe, pre World War II \citep{polanyi36}. By committing to central plans in the pursuit of knowledge, ``thinkers'' (scientists, scholars, intellectuals) relinquished their power to contradict the powerful by appealing to the truth. Judging by what was to become the immediate political state of Europe, Polanyi's arguments proved to be eerily accurate.

Interestingly, it is not hard to notice that Polanyi's arguments as applied to the production of knowledge bear some similarities with Milton Friedman's theories of economics, particularly his opinions about the necessity of independence in research endeavours, which strikingly resemble the \textit{laissez-faire} attitude that generally characterised Friedman's well-publicised work. Indeed, others have noticed and written about such similarities. \citet{roberts99}, for instance, boldly maintain that Polanyi not only reached Friedman's conclusions two decades in advance, but that, had his (mostly unnoticed) economic theories been acknowledged and applied, not only would he have ``eclipsed both Keynes and Friedman by his early synthesis'' but also ``[e]conomics and public policy would have been spared the long and pointless Keynesian odyssey towards big government'' \citep[pp.575--576]{roberts99}. While this last statement may sound somewhat of an exaggeration, particularly after witnessing the effects of the 2008 financial crisis, it is safe to say that Polanyi did indeed anticipate some of Milton Friedman's conclusions by over two decades.

Embodying the archetype of the renaissance man, Polanyi moved rather comfortably between many disciplines. Nevertheless, by the time he wrote \textit{Full Employment and Free Trade} \citep{polanyi45}, he lacked the professional prestige as an economist to be taken seriously outside purely academic environments. Despite this, he understood that many of his views on epistemology very well fit within economics theory at large. By the time he wrote \textit{Full Employment and Free Trade}, the Keynesian viewpoint that full employment was not a natural condition of a free market economy was already held as doctrine by spheres of power and prevailing academics. Polanyi contradicted this statement, arguing that money itself was at the heart of the problem. While he agreed with Keynes that unemployment was a result of lack of ``total aggregate demand'', he took a retrospectively logical yet bold next step, arguing that lack of demand was caused by lack of circulating money, which, injected to the economy in reasonable amounts, he argued, would not only bring about full employment but also \emph{not} become inflationary. In this, he viewed a mostly unregulated free market as a means for achieving this purpose.

In the context of this thesis, the noteworthiness of Polanyi's work in economic science lies in the fact that it sheds light on the versatility of the man and his reasoning, by demonstrating just how far-reaching his conclusions would become and exemplifying how they may be re-interpreted and used in other fields and other topics, such as education and new media studies, making him relevant well into the information age. As such, the next section outlines the specific implications of Polanyi's work as they apply to Hackers, the dawn of hackerspaces and the more recent DIY movements of communities, cultures and projects stemming from the world-wide-web.

\subsection{Polanyi the Hacker}

To fully grasp the implications of Polanyi's work within the context of this thesis, they must first be considered amidst the wider spectrum of the educational field. 

Although Polanyi himself dealt with the philosophical and scientific repercussions of his work rather than the educational ones, it is not difficult to mentally transpose such implications to the field of education in a broad and general way. Nevertheless, a thorough and complete analysis is necessary ---a void that has been filled in part by \citet{kane84}, whose work deals with this topic.

Education, argues Kane, cannot and should not look down on knowledge that can be described as subsidiary, for it is vital to the success of the learning process, serving a double purpose without which efforts in this area would be all but futile. First, subsidiary knowledge has the potential to provide the student (or trainee or, simply, learner) with a foundation upon which to acquire information and make judgements, even if this foundation lacks the solidity of proven or more widely accepted theories or pieces of common knowledge. Fermat's last theorem can be viewed as one such of those pieces of subsidiary knowledge. This algebraic conjecture\footnote{Brief explanation of the theorem} went unproven for the better part of four centuries, yet it provided a well-anchored foundation upon which modern algebraic theory was (and still is) considerably based. Not only did its formulation significantly contribute to the development of mathematics, but its adoption ---prior to its final proof--- as almost a scientific fact allowed many generations of mathematicians to advance modern algebra and geometry to the point where we find them today.

Second, subsidiary knowledge has the potential to become the basis of its own self-destruction. Given an ideal environment of free enquiry, students may dispose of presented knowledge as a result of their own research, assumptions or conclusions, thus shifting between paradigms or bringing new ones into existence. For this phenomenon to take place, students must be encouraged to exercise independent thought and feel free to contradict the teacher or mentor. While it is true that inquisitive questioning has been largely encouraged in western educational systems, particularly since the introduction of compulsory mass schooling \citep{dillon88}, the type of encouragement of critical reasoning (even towards authority figures) seen by Polanyi as a facilitator in the process of knowing is, to this day, scarcely found, particularly in the earlier stages of the student's development, when the inculcation of discipline and adherence to social norms take precedence over free enquiry. \citet[p.239]{kane84} illustrates the reason for the necessity of such mechanisms in terms of what he calls a student's ``vision of reality'':

\begin{quote}
\ldots as each new addition to knowledge transforms our cultural presuppositional vision of reality, so each new lesson a student learns transforms his vision\ldots A predispositional shift occurs with each act of learning. The latent quality of the shift, the ability to reinterpret experiences and make new and varied situations comprehensible is present in the individual as a heuristic vision. It identifies new areas of knowledge and oversees future intellectual pursuits.
\end{quote}

These two factors combined, reliance and potential for ``destruction'' of subsidiary knowledge, make its transmission and nourishing essential for achieving a healthy learning environment. Yet, an analytical reflection on the validity of this approach as it compares to positivistic views praising total objectivity is still necessary. As Polanyi sees it, complete objectivity should not be sought only because it is unattainable in practice, but because it is, in fact, detrimental to the learning process, as it becomes an act of wilful ignorance of a substantial component of what is to be known. Despite this, full objectivity remains an ideal highly praised and sought after by many in the world of academia. \citet{lavoie95}, for instance, argues in favour of the objective ideal, maintaining that it is ``a demanding moral criterion'' that embodies ``the fundamental ideal of higher education''. He is one of many scholars [enter refs] who still perceive objectivity (in Polanyi's sense of the word) as a sort of holy grail the educational process should aim for, even if some of them concede that in practice such concept is unachievable.

The fundamental flaw in the arguments of those who advocate positivistic objectivity lies in the loss of heuristic approaches that accompany subsidiary knowledge and without which it is extremely difficult (if not impossible) to succeed in imparting education (or more generically, transferring knowledge and skills). \citet{collins01} has demonstrated the need for rule-of-thumb and non-verbal practices in real-world scenarios, by detailing how western scientists were unable to reproduce a scientific experiment performed by their Russian counterparts for over twenty years, resulting in frustration and a high degree of scepticism with regards to long-claimed success by the Russians. Despite abundant literature and resources, it was only through close cooperation with the Soviet scientists and their use of highly unorthodox, yet valid techniques that the western team finally managed to reach the same results.

The main implication of the need for subsidiary knowledge in education is that no amount of written or oral information will match the effects of a hands-on approach, allowing the student to freely experiment and sometimes diverge from the central line that he or she is being introduced to. Curricula, thus, require a certain degree of flexibility that encourages students to pursue their own interests, obviously within an acceptable framework. Such approach would doubtlessly bring formal educational environments closer to the essence of what a hackerspace represents. It is, then, no coincidence that the process of learning that takes place inside hackerspaces adheres to Polanyi's vision ---not by plan, but rather as a natural result of the atmospheres that characterise such places.

Hackerspaces.org, currently the largest, most comprehensive source of information about hackerspaces, defines them as ``community operated physical places, where people can meet and work on their projects''. While there is no particular underlying topic unique to all hackerspaces, most of them gravitate around computer technology. As will be discussed in section [ref], they differ from traditional educational settings in a number of ways, most interestingly in the fact that they are ``user-led'' and informal by nature. How, then, is a hackerspace any different from a traditional school with regards to the application of Polanyi's precepts? Perhaps more importantly, why are such precepts particularly well-suited for their study?

A preliminary answer to these questions can be provided by comparing Polanyi's ideas as they apply to education to those of American educational theorist John Dewey. Polanyi and Dewey were ideologically close with regards to several issues, notably the necessity for a ``hands-on'' approach to education, where the student is not to be treated as a passive participant, meant to absorb notions and concepts but rather as an active individual whose permanent feedback and \emph{social} contact is crucial for the process to become successful. Dewey developed the concept of ``practitioners as inquirers'', whereby \textit{those who do} intuitively and usually effortlessly learn from their activities, in what \citet{evans00} describes as an ``active engagement with nature''.

Dewey saw experience as a necessary basis for knowing, one which the subject employed in order to be able to actively engage with the activity that brought the knowledge. He described experience as ``a process of undergoing; a process of standing something; of suffering and passion, of affection, in the literal sense of these words'' \citep{dewey81}. Similarly, as discussed earlier, Polanyi's concept of tacit knowing relies heavily on knowledge through practice. Yet, Polanyi went one step further, noting that explicit (focal) knowledge should derive from the practical, subsidiary act. In other words, Dewey would be satisfied by the act itself, whereas Polanyi would note that there should be an explicit moral to that act. Polanyi's views provide a remarkable match to the hacker ethic ---in this case to the notion of the ``hands-on'' imperative as described in section [ref]. As will be demonstrated, hackers regard practical work as a vessel through which they can learn.

% ^ elaborate on the hands-on approach (2)

Both Dewey and Polanyi strongly advocated academic freedom. As discussed earlier, Polanyi was terrified by the Soviet idea of centrally-imposed research guidelines, which, in his eyes ``denied any grounds for claiming freedom of thought''. So too, did Dewey, as evidenced by his entry into the International League of Academic Freedom in 1935, notably the same year in which Polanyi had his conversation with Bukharin, the Soviet thinker.

% ^ elaborate on academic freedom (2.5)
Furthermore, both Dewey and Polanyi coincided, to a certain degree, in the necessity of the student to be able to pursue his or her interests \citep{schon92}. Yet Dewey held somewhat more conservative views with regards to students' freedoms. While he maintained that the idea of having passive students solely commit to absorbing pre-established curricula was indeed inadequate, he also suggested that too much reliance on students' motivations would contribute in swaying them away from adequate curricula. Polanyi, on the other hand, viewed the learning process as a much more ``aimless'' endeavour (at least initially) that allowed students to find points of interest through their own discovery. 

Academic freedom for students, thus, became closely tied to the concept of purpose or objective. Dewey argued for the presence of a sense of purpose driving student activities, ---something to make such activities meaningful enough for the process of knowledge acquisition to be successful[cite]. Polanyi, on the other hand, equated initial lack of purpose with the same aimlessness as noted earlier ---a stage full of potential, or an ``empty focus'' that is to become interesting by a search for logical coherence. In fact, Polanyi specifically maintained that no practical application of the knowledge gained was to be necessary. In doing so, he heavily criticised systems such as that of the Soviet Union, where applied research was seen as the single unique way to benefit from the pursuit of knowledge. Dewey's plea for academic independence, then, was not as radical as Polanyi's. 

Under the latter's scheme, motivation for learning would come not from a specific desire to contribute to society (which is not to say that it does not happen. It certainly does, but rather as something of a side-effect) but for a varied number of reasons, one being the self-gratification gained from acquiring knowledge for its own sake, which, as will be demonstrated later, is commonplace amongst hackers and typical of the flexible and lenient atmospheres of hackerspaces. Innovation, in this context, is partly (and remarkably) driven by pleasure. 

Student freedom and a general sense of seeming purposelessness closely fit the ethos of Steven Levy's hacker ethic. The third precept ---``Mistrust authority, promote decentralization'' can be seen as conforming to this general approach. The horizontal learning structure found in hackerspaces, where those who learn often cannot be told apart from those who teach is exactly the type of ``primal soup'' environment, one full of potential, where new knowledge is acquired by means of human interaction, reliance on subsidiary knowledge (as well as focal knowledge) and the prevalence of a generally inquisitive environment created by lack of a strict sense of purpose.


% elaboration of non-authoritarian (3) and horizontal relationship (4)  and de-centralisation (1)

Furthermore, Polanyi's flatter educational structure deprives teachers from wearing a common veil ---that of authority figures. Rather, they are perceived as peers who can pass on tacit and explicit knowledge. Steven Levy's fourth principle of the Hacker Ethic states that ``hackers should be judged by their hacking, not by bogus criteria such as degrees, age, race, sex, or position''. By overseeing foreign social conventions from their environments, hackers immerse themselves inside cultural spheres that permeate that which is taught. As will be shown in the next chapter, hacker culture is the product of several decades of interactions, relationships and interplay amongst members of a subculture rich in philosophical principles, rituals and traditions, where prestige comes not from formal titles but from non-confrontational displays of aptitude and acumen, which bring subjects closer to the core of closely-knit groups. By proving his or her worth, the hacker is further and more intimately accepted into the group. Polanyi elaborates:

\begin{quote}
This assimilation of great systems of articulate lore by novices of various grades is made possible only by a \emph{previous act of affiliation}, by which the novice accepts apprenticeship to a community which cultivates this lore, appreciates its values and strives to act by its standards. \ldots Just as children learn to speak by assuming that the words used in their presence mean something, so throughout the whole range of cultural apprenticeship the intellectual junior's craving to understand the doings and sayings of his intellectual superiors assumes that what they are doing and saying has a hidden meaning which, when discovered, will be found satisfying to some extent. \citep[p.207]{polanyi58}
\end{quote}

As human beings, indissolubly tied to their cultural environments, those who teach, tacitly but inexorably transmit subsidiary knowledge via personal judgements. Presential environments thus expose themselves as much more suitable settings for the transmission of certain skills. In contrast, virtual environments such as those made possible through the Internet (forums, chat rooms, boards, etc.) simply cannot convey the entire range of information passed on by the teacher to his or her student in the form of physical contact. This is not to say that knowledge cannot be transmitted virtually but rather that given the range of activities performed in hackerspaces, physical environments are considerably better suited for such purposes. Furthermore, since presential and virtual are by no means mutually exclusive, hackerspaces benefit from the best of both worlds, by building strong local communities while keeping close contact with peers from around the world.


% Polanyi's epistemology as it relates to the hacker ethic

% 1. De-centralisation: Means that, just as polanyi sees centralised or bureaucratised science as an abomination, so do Hackers hold contempt for such phenomena

% 2. Knowledge through practice: hackerspaces experiment first and rationalise later. This means not only that practice is a key element but also that subsidiary knowledge runs free
% 2.5 Knowledge for the sake of knowledge. Science makes its most important leaps when it is allowed to experiment and mingle without a specific aim or result in mind. Same goes for the hacking for hacking's sake ethos.
% 3. Knowledge is non-authoritatian, in fact, under Polanyi, subsidiary knowledge is there to be challenged. In this sense, it equates to no 'bogus criteria'
% 4. The relationship between learner and teacher is quite horizontal, thus equating to the peer-cooperative athmosphere found at hackerspaces and bohemian joints 

%In that sense, polanyi's principles equate to the hacker ethic in the sense that: decentralisation, hands-on approach<=>experimentation , knowing as 'relying on' which equates to free knowledge. Also, knowledge is subjective which is non-technocratic, meaning hacker and bohemian-friendly approach.
