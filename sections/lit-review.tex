

\section{Coming of Age of a Subculture}



At first glance, a clear evolutionary link between the hacker subculture and bohemians, as first described in the early 19th Century is dim. In studying the history of Australia's bohemian tradition, Tony Moore \citeyearpar{moore07} analysed the relationship between 19th Century-style bohemians and the multitude of subcultural and countercultural movements born with the generation of baby boomers, from the late 1950s on. While technologically-minded hackers may be at odds with \citepos{roszak69} critique of the state of the society he lived in, which he described as being a harmful, or rather, dulling \textit{technocracy}, many of the bohemian traits identified by Moore seem as compatible with the early MIT hacker ethos as they were with those of the writers, artists and activists that inherited the values and customs of the bohemian legacy. In this chapter I argue that hackers are indeed unlikely heirs of that legacy too. I also analyse how the nuances and cultural traits of their origins manifest in the present, leading to the origin of modern-day hackerspaces.

\label{history}
\subsection{Hacker Origins and Philosophy}

The tradition of hacking has its origins within a formal academic environment. Indeed, the word ``hacker'' as a descriptor of the creative, technologi\-cal\-ly-savvy and computer enthusiasts first became used at the Mas\-sa\-chu\-setts Institute of Technology, where it derived from the older tradition of harmless and ingenious pranks some students devised (and still devise) at the University's campus \citep{levy84}\footnote{\textit{Hacking}, in this context, is a wonderful MIT tradition whereby students come up with elaborate jokes that demonstrate their technical prowess as well as their sense of humour. One example of a memorable hack was the placement of a police car on the top of the Institute's \textit{Great Dome}. For a full list of hacks, see \texttt{http://hacks.mit.edu/}.}.

As a result of the media-fuelled adoption of the term as a synonym for computer criminal, its meaning within academic boundaries is heterogeneous at best and discordant at worst. Some scholars and researchers in the field of information technology and computer security have chosen to follow the media's trend, using the term to allude to malicious cyber-intruders, thus making literature on the subculture that originated at MIT in the late 1950s extremely difficult to distinguish.

In his book \textit{Hacker Culture}, author Douglas Thomas \citeyearpar{thomas02} takes a rather neutral stance on the etymological divergences of the word, choosing instead to focus on the consequences of such divergences: ``the very definition of the term `hacker' is widely and fiercely disputed by both critics and participants in the computer underground''. In his view, this ``gives a clue to both the significance and the mercurial nature of the subculture itself''. Such eclectic positions, however, add an additional layer of complexity to the task of performing a thorough literature review on \textit{any} of the word's conveyed meanings.

Indeed, the concept of hackers as a collective is filled with almost as many contradictions as the word itself. While it would be easy to place them on the more formal side of Theodore Roszak's \citeyearpar{roszak69} cultural division of the post-war years, as members of a closed inner circle, or, more aptly, a techno-elite, I aim to prove that, in fact, hackers are quite the opposite, having more in common with Roszak's counterculture. Other scholars, such as McKenzie Wark \citeyearpar{wark04} have gone as far as describing hackers as a digital age Marx-style \textit{proletariat} where ``vectiorialists'' (capitalists) control and exploit their labour and profit from it.

Steven Levy's \textit{Hackers: Heroes of the Computer Revolution} \citeyearpar{levy84} is widely recognised as a fundamental, landmark piece of literature when it comes to historically documenting the emergence of the original hackers as an identifiable group and subculture. While not academic in nature, Levy's work can not only be considered an obligatory reference but also a document of historical significance itself, having provided an initial written declaration of philosophical principles, to which he dedicated an entire chapter of his book. Levy coined the term `The Hacker Ethic' to describe set of undeclared maxims that seemed to have originated in parallel with the movement itself, when early hackers lurked building 26 at MIT, hoping to harness unused, idle time from one of the first (and very primitive) computers ever assembled:

% In the mid-’80s, following a rash of computer break-ins by teenagers with personal computers, true hackers stood by in horror as the general public began to equate the word — their word — with people who used computers not as instruments of innovation and creation but as tools of thievery and surveillance levy2010


\begin{quotation}
``\ldots the dozen or so hackers were reluctant to acknowledge that their tiny society, on intimate terms with [the computer], had been slowly and implicitly piecing together a body of concepts, beliefs and mores.

The precepts of this revolutionary Hacker Ethic were not so much debated as silently agreed upon. No manifestos were issued. No missionaries tried to gather converts.''
\end{quotation}

\noindent
If not the first, Levy was certainly amongst the earliest theorists who saw the necessity to explicitly declare those ``concepts, beliefs and mores'', \textit{The Hacker Ethic}, consisting of six precepts:

\begin{enumerate}
\item Access to computers ---and anything which might teach you something about the way the world works--- should be unlimited and total. Always yield to the Hands-On Imperative!
\item All information should be free
\item Mistrust Authority - Promote Decentralization
\item Hackers should be judged by their hacking, not bogus criteria such as degrees, age, race or position
\item You can create art and beauty in a computer
\item Computers can change your life for the better
\end{enumerate}

\noindent
Inherent in such principles were ideas of freedom of information, contempt for the establishment and technological determinism. I suggest that the Hacker Ethic provides a clear, coherent basis upon which to perform an analytical dissection of the emergence of the hacker movement. Precepts 1 and 2 summarise the movement's views on issues of freedom, with an emphasis on information as well as their opposition to restrictions imposed by copyright and patent law.

Hackers were born within ---and still share close ties to--- academia. Across the world, universities still foster and cherish the sharing of information. Gift economies are prevalent amongst academics, who continually build upon each other's work and who, just like those first hackers, are flattered ---not threatened--- to see others use and expand their own work.

Precepts 3 to 5 present an interesting perspective on Hackers' world-views, particularly when it comes to their relation with the wider spectrum of society. By making a personal interpretation of these three principles, I conclude that today's Hackers are heirs to the wider bohemian tradition that begun in Europe during the 19th Century and subsequently evolved into the 1960s and 1970s countercultural movements, with which they shared similar views, tastes and political ideas.

Lastly, I judge the sixth precept to represent a growing trend towards technological determinism, reflective of the growing view that human and social development are closely tied to the advance of technology. 
%insert very very brief intro to tech determinism and comment out for paper


\subsection{Heirs of the Bohemian Tradition}

% --------------------------------------
% SUBINDEX
% --------------------------------------

% 1. Multiplicity of countercultures
% 2. Denial or contempt for the establishment
% 3. Criminal label & other carnivalesque traits
% 4. Geeks and Nerds
% 5. Position towards commercialism and capitalism
% 6. Nostalgia




% --------------------------------------
% Begin subsection
% --------------------------------------


% % 1. Multiplicity of countercultures
\subsubsection{One Counterculture, Many Subcultures}


In 1969, Theodore Roszak described the growing rebelliousness of the baby boomer generation as a \textit{counter-culture}: ``a culture so radically disaffiliated from the mainstream assumptions of our society that it scarcely looks to many as a culture at all, but takes on the alarming appearance of a barbaric intrusion''. Such disaffiliation was certainly not new: 19th Century bohemians (and others before them) already held many of the social grievances described by Roszac. It was, however, the sheer scale of the new movements ---strengthened in numbers by the offspring of couples formed in the post-war years--- that made the 1960s manifestations particularly powerful, even if, in reality, most of those movements were heterogeneous in nature.

As argued by various scholars \citep{mcgregor75,spates76,eder90}, a single, uniform counterculture as proposed by \citeauthor{roszak69} was nothing more than a gross generalisation. \citet[p.148]{hebdige87} further argued that the term came to represent an ``amalgam of
`alternative' middle-class youth cultures'' in macro-political opposition to what has been called the ``the establishment'', ``straight society'' or simply, ``the mainstream''. Yet others, like \citet[p.87--88]{marchant03} have criticised this view as over simplifying.

In this context, I take a practical approach and view individual movements as different, specific subcultures, all of which are enclaved in a countercultural social scope. Therefore, as eclectic as such groups were, opposition (whether backed by micro or macro-political action) was indeed their strongest bond. In this sense, hackers, bohemians, students and hippies were no different. As will be argued, such disdain for established traditions, authorities and the prevailing social order offered them empowerment yet came with a sense of irony, as many of the movements (hackers included) were born either inside or as a result of established institutions. Hackers' genesis, for instance, took place inside universities and in many cases with the financial and technical support of much-loathed institutions like the U.S. Department of Defence.

% ****************************************
% This is where the clarification on the
% issue of subculture/counterculture needs
% to go. Get subcultural reader
% ****************************************



% 2. Denial or contempt for the establishment

\subsubsection{Contempt for the Establishment}

Despite this element of irony, denial of established values, institutions and customs became the driving force for the self-identification of all movements of the time. In his Doctoral Thesis, Tony Moore examined the cultural \textit{status quo} that bohemians before them both challenged and benefited from:

\begin{quote}
\ldots bohemians performed publicly an imagined or hoped for personal autonomy from art markets that involved style, behaviour, art, social formations and even politics that transgressed and subverted, \emph{but never overturned}, bourgeois society \citep[p.10]{moore07}. [emphasis added]
\end{quote}

\noindent

By publicly expressing contempt for that society and positioning themselves as social outsiders, such characters ---who had long before acquired a sense of group identity--- managed to leverage a sense of ``forbidden fascination'' exerted in the wider social strata by means of ``coded fashion and recreation that could be read by the initiated'' \citep[p.17]{moore07}. Upon the dawn of the 1960s movements, such codes were further disseminated and strengthened by the powerful effects of the mass media.

As political awareness increasingly grew amongst many subcultural movements, condemnation of mainstream values became not only more outspoken but action-driven, shifting from a rather tolerant acceptance of the prevailing \textit{status-quo} towards a more politicised resistance that culminated in the protests of May of 1968 in France, the violent, anti-Vi\-et\-nam intervention demonstrations in university campuses across North America and the radicalisation of many movements across Latin America, where the very real possibility of the overturning of bourgeois society spurred revolutions and dictatorships whose consequences are felt to this day.

Drawing parallels between these events and the emergence of hackerism is not a simple task ---one not made any less complicated by the fact that the amount of scholarly works on the subject is rather modest, particularly when compared to other movements of the time. The limited amount of resources available nevertheless sheds some light on the issue of hackers' analogous disdain for established orders. \citet{levy84}, for instance, provided a sound starting point with his Hacker Ethic. The third maxim, ``Mistrust authority. Promote decentralization'', sheds extra light on the link between hackers and the wider countercultural spectrum.

This link becomes clearer by examining the history of the dawn of hackers, whose rhetoric, like that of many other movements of the time, became increasingly politicised during the 1960s and 1970s. Hackers imagined a world ``where computers would lead the way to a new, liberating lifestyle'' \citep[p.168]{levy84}. Much like other movements, they envisioned a way out of the social constraints of society, yet did so in their own, unique way, believing computers to be the main vehicle with which to achieve this.

In addition, \textit{Hackers} bluntly documents behaviour that is quite contrary to Roszak's concept of ``technocracy''. MIT hackers, Levy argued strongly, held passionate and vocal disdain for Weber-style bureaucracy. The focus of such disdain was inevitably embodied by International Business Machines (IBM) ---the devil they knew best, and later by countless more companies as well as the corporate culture they seemed to represent. In describing hackers' loathing for IBM types, Levy commented:

\begin{quote}
All you had to do was look at someone in the IBM world, and note the button-down white shirt, the neatly pinned black tie, the hair carefully held in place, and the tray of punchcards in hand. \citep[p.42]{levy84}
\end{quote}

Interestingly, this exact same stereotype is referenced by Roszak in his book. Upon discussing the nature of ``grown-upness'' he describes countercultural youth as having ``better ideas than GM or IBM seem able to offer'', describing them as a ``scruffy, uncouth, and often half-mad lot'' \citep[p.32]{roszak69}. Hackers' choice of clothes and odd sense of style thus followed a conscious effort to distance themselves from the formality of the corporate environment, embodied by the stereotype of the IBM employee, in the same sense as hippies and others rebelled themselves by opposing the idiosyncrasies and fashions of 1950s suburban America \citep[p.34]{heath05}. Their anti-establishment sentiments differed, however, in one fundamental way when compared to those of other subcultural groups of the time: their inherent motivations.

Beats, hippies and even anti-war protesters' unease seemed to stem from evolving views on Marxist and Freudian theories. Countercultural interpretations of Freud saw culture, or rather, civilisation as a sort of straitjacket constraining individuality and freedom \citep{heath05}. Hackers instead viewed individual freedom as achievable \emph{through} the more pressing matter of freedom of information. Thus, just as individuality was crucial for the development of group identity and sense of belonging in the case of the former, freedom of information became hackers' foremost foundational dogma. In practice, no single subculture would have thrived in its absence.



% 3. Criminal label & other carnivalesque traits

\subsubsection{The Criminal and the Carnivalesque}

Hackers are curious by nature. They satisfy their curiosity by engaging in acts of discovery, whether that means solving a problem or deciphering the inner workings of a machine or network. While, every so often, a hacker may break the law, there seems to be a wide mis-representation of hackers ---true hackers--- as criminals. This negative connotation has its origins some time in the 1980s. Richard Stallman \citeyearpar{stallman02b} explains: 

\begin{quote}

\ldots when I say I am a hacker, people often think I am making a naughty admission, presenting myself specifically as a security breaker. How did this confusion develop?

Around 1980, when the news media took notice of hackers, they fixated on one narrow aspect of real hacking: the security breaking which some hackers occasionally did. They ignored all the rest of hacking, and took the term to mean breaking security, no more and no less. The media have since spread that definition, disregarding our attempts to correct them. As a result, most people have a mistaken idea of what we hackers actually do and what we think.

\end{quote}

While the media may have been responsible for spreading the negative connotation, validation for the practice seems to have been the result of the publishing of three landmark books about computer criminals \citep[p.xiii]{thomas02}. Katie Hafner and John Markoff's \textit{Cyberpunk} \citeyearpar{hafner91}, William Gibson's iconic novel, \textit{Neuromancer} \citeyearpar{gibson84} and Clifford Stoll's \textit{The Cuckoo's Egg} \citeyearpar{stoll89}. All three books referred to their characters as ``hackers'', making little effort to draw a clear semantic distinction between their respective characters (real and fictional) and the wider hacker subculture. 

Stoll, a scholar, mathematician and astrophysicist, proficient in several computer languages and technologies was aware of the emerging trend and the irritation it caused amongst self-confessed hackers, yet chose to follow it. ``A computer wizard?'', he asked sarcastically at the beginning of his book. ``Not me ---I'm an astronomer'' \citep[p. 1]{stoll89}.

Perhaps as a wilful effort to appease the outcry, Stoll justified his decision with an apologetic opening paragraph:

\begin{quote}
What word describes someone who breaks into computers? Old style software wizards are proud to be called hackers and resent the scofflaws who have appropriated the word. On the networks, wizards refer to these hoodlums of our electronic age as ``crackers'' or ``cyberpunks''. In the Netherlands, there's the term ``computervredebreuk'' ---literally, computer peace disturbance. Me? The idea of a vandal breaking into my computer system makes me think of words like ``varmint'', ``reprobate'' and ``swine''. \citep[p.11]{stoll89}
\end{quote}

Throughout his book, Steven Levy provides a detailed rationale for hackers' contempt for security, arguing that it lied not in a desire to break the law, steal or cause mayhem, but rather as a result of common beliefs about the free flow of information and its benefits. Richard Stallman articulately summarised this mantra:

\begin{quote}
Hackers typically had little respect for the silly rules that administrators like to impose, so they looked for ways around. For instance, when computers at MIT started to have "security" (that is, restrictions on what users could do), some hackers found clever ways to bypass the security, partly so they could use the computers freely, and partly just for the sake of cleverness (hacking does not need to be useful). (\ldots)

In the hacker's paradise, the glory days of the Artificial Intelligence Lab, there was no security breaking, because there was no security to break. 
\end{quote}


To those original hackers, some intrusions were not only morally justified, they were necessary. Their motivations were driven by the satisfaction of a legitimate thirst for knowledge. None of the original hackers would have considered exploiting systems for profit or malice: their goal was to eliminate ``the odious concept of passwords'', arguing that systems ---and the underlying technology behind them--- belonged ``not to the author but to all users of the machine'' \citep[p.127]{levy84}, echoing a socialist rhetoric.

%\footnote{Mark, Tony: I'm looking into relating this with Marxist views on property. What do you think?}.
%relate this to marx and property

Early bohemians suffered from the same predicaments, albeit on a different scale and with notably different consequences. Marx, argues Tony Moore, used the term ``\textit{le boh\`{e}me}'' to refer to a ``vagabond character, with a strong connotation of poverty and even criminality''. To some degree, Marx mis-understood the bohemian ideology and lifestyle.

%elaborate on bohemians (or hippies') connotations of criminality

Yet, it seems that, in some cases, notions of criminality were at least partly self-imposed: vehicles for ``acting out'' a part that would wilfully separate them and their prospects from ``more conventional artists'' \citep[p.30]{moore07}. By engaging in \textit{carnivalesque} practices that would go on to become collective rituals, bohemians embraced the power that came out of their cultural transformation. Many of these rituals and attitudes also became suitable targets for media speculation, resulting in simplified and often imprecise portrayals, which, as with hackers, lead to the formation of simple stereotypes.

This process also facilitated mechanisms of cultural ``co-optation'', by which images, accessories and garments were stripped of their cultural significance to be sold for profit \citep{heath05}. Subcultural movements from the 1960s and 1970s hence unwillingly facilitated the massification of a consumer culture that borrowed and appropriated whatever could be marketable about them. 

Early hackerism lacked the histrionic sophistication required to engage in carnivalesque displays. Driven by more pragmatic principles, the hacker subculture saw no benefit from the acquisition of symbolic capital beyond the boundaries of its own, narrow field of interest. Hackers did not wish to spread their ideas by ``converting'' anyone, nor did they aspire for recognition beyond their small circles of peers. All this led to their characterisation as reticent, isolated and non-social. The Jargon File\footnote{The \textit{Jargon File} is another landmark document in hacker history. The multi-authored text file was first started some time in the late 1970s. Originally conceived as a sort of dictionary to describe hacker slang, the Jargon File became a living document ---a historical account of the movement, written by hackers themselves. While difficult to cite and interpret because of its many authors and versions, the Jargon File does have historical value as a witness to the birth of the movement. A polished, organised and citable version of the file was originally published as a book in 1983 by Guy Steele and re-published in the early 1990s by Eric S. Raymond. The file (and by extension, the book) provides a window to early hacker culture as seen by themselves. In that sense, it is a priceless historical resource. In this work, the Jargon File shall refer to Steele's and Raymond's \textit{The new hacker's dictionary} \citeyearpar{raymond93}.}, written in the third person, presents a view that seems to be diametrically opposed to images of flamboyant bohemians and artists:

\begin{quote}
Hackers have relatively little ability to identify emotionally with other people. This may be because hackers generally aren't much like `other people'. Unsurprisingly, hackers also tend towards self-absorption, intellectual arrogance, and impatience with people and tasks perceived to be wasting their time. \citep[p.743]{raymond93}
\end{quote}

From this perspective, it is hard to equate hackers with sexually liberated, free-loving hippies, descendants of the dandies or fl\^{a}neurs of yes\-ter\-year. Yet, things have changed. In spite of this apparent contradiction, the hacker subculture has greatly evolved in recent years. The result has led to a shift media portrayals and a more benevolent general perception, albeit under the more general and vague ``geek'' label.

% 4. Geeks and Nerds
\subsubsection{Revenge of the Nerds: A Tale of Co-optation} %or cultural appropriation

As the meaning of the word \textit{hacker} shifted as a result of the influence of the mass media \citep{kinney93}, the original nuances associated with it tended to shift towards other words, more vague in definition and, initially, charged with a high negative overtone, namely ``nerd'' and ``geek''. Steven Levy noticed this shift upon re-visiting his book in 2010:

\begin{quote}
The kind of hacker I wrote about was motivated by the desire to learn and build, not steal and destroy. On the positive side of the ledger, this friendly hacker type has also become a cultural icon ---the fuzzy, genial whiz kid who wields a keyboard to get Jack Bauer out of a jam, or the brainy billionaire in a T-shirt--- even if today he's more likely to be called a geek. \citep{levy10}
\end{quote}

It can be argued that, today, the essential meaning of all three words, ``hacker'', ``nerd'' and ``geek'' is closely intertwined, much like a Venn diagram, where large areas of each seamlessly intersect. Yet, nuances and applications give each a unique set of values that ultimately convey similar but particular concepts, having co-evolved during the past three decades. \citet{kendall99} notes that ``Nerd'' became a relatively common term in ``TV shows about teens'' since at least the mid-1970s, yet, it was only during the mid-1980s when it came to describe people ``overly involved with, and skilled in the use of computers''. This is precisely the time when the original meaning of the word ``hacker'' began to shift.

Similarly, \citeauthor{kendall99} notes the relative closeness of ``geek'' and ``nerd'', while noting that the former has a much less negative connotation. To make her point, Kendall cites a character Douglas Coupland's novel, \textit{Microserfs} \citeyearpar{coupland95}, who provides an interesting view of the differences between the two: ``geek implies hireability, whereas nerd doesn't necessarily mean your skills are 100 percent sellable. Geek implies wealth''. To the uninitiated, however, the two differ starkly from ``hacker''.

Writers of the Jargon File, on the other hand, were quite aware of the nuances and subtleties with regard to the meaning of each of the three words, as well as their points of intersection, as demonstrated by the following quote:

\begin{quote}
\ldots many hackers have difficulties maintaining stable relationships. At worst, they can produce the classic geek: withdrawn, relationally incompetent, sexually frustrated, and desperately unhappy when not submerged in his or her craft. Fortunately, this extreme is far less common than mainstream folklore paints it — but almost all hackers will recognize something of themselves in [the stereotype] \citep[p.744]{raymond93}.
\end{quote}

In recent years, however, the \textit{geek} label has become an interesting example of linguistic and social reclamation, one that has come to be a source of pride and, increasingly, even ``coolness'', in a way that is analogous to what happened with other pejorative terms and the collectivities they described, such as ``queer'' and ``gay'' in the case of homosexuals \citep{brontsema04}.

Building upon Bourdieu's concept of \textit{cultural capital}, Sarah Thornton \citeyearpar{thornton96} coined the term \textit{subcultural capital}, defining it as the ``hipness'' that operates outside Bourdieu's main \textit{fields}, but that rather thrives within ``less privileged domains'' \citep[p. 11--14]{thornton96}. The author argued that, while not as easily convertible into other forms of capital (economic, for instance\footnote{Thornton does provide examples in which her subcultural capital as the \textit{hipness} factor, can be converted into economic capital such as the case of DJs or fashion designers.}) and unbound by class, subcultural capital is gained from performing selective consumption and engaging in media appropriation, particularly amongst younger people. The growing (but still niche) market that has risen for goods that link their consumers with the geek label (``geek cred'') in the form of cryptic messages and references only interpretable by a selected few \citep{tocci07}, is a prime example of this trend.

While still in its infancy, such development is no different from the commercial frenzy that followed cultural appropriation processes stemming from countercultural movements of the 1960s and 1970s, resulting in fruitful commercial endeavours. According to \citet[p.91]{moore07}, a ``loyal countercultural market'' was swiftly exploited by savvy entrepreneurs who understood that the image they projected ---one of individualism and au\-then\-ticity--- had tremendous commercial potential. \citet{heath05} suggest that, following such processes of commercial co-optation, the core principles of these movements were overshadowed by their aesthetics:

\begin{quote}
\ldots the hipster, cooling his heels in a jazz club, comes to be seen as a more profound critic of modern society than the civil rights activist working to enlist voters or the feminist politician campaigning for a constitutional amendment \citep[p.32]{heath05}.
\end{quote}

The \textit{faux} link between countercultures, they argued, along with Freud's concept of \textit{eros} and ``the revolution'' became extremely beneficial for marketers and capitalists of all shapes and sizes, all eager to benefit from whatever was considered to be in opposition to the so-called mainstream.

The trend was predictably reflected in the media. The Broadway musical \textit{Hair} became a massive success when brought to Sydney in 1969, benefiting from the society-wide appeal with the lure of the \textit{hippie} ideology of sex, drugs, music \emph{and} social revolution \citep[p.104]{moore07}. Likewise, today's ``geek-oriented'' audiovisual works generate enormous amounts of interest. Situation comedies such as \textit{The Big Bang Theory} in the United States and \textit{The I.T Crowd} in the United Kingdom became commercial hits \citep{rodman09,smith06}.

The ``geek chic'' phenomenon thus seems to be experiencing the same kind of co-optation processes by means of intervention by the entertainment industrial muscle, while the subcultural values remain alive online and in small groups such as those created by hackerspaces. Having a strong record of participation online separates the wannabes from the real hackers or geeks \citep{tocci07}. Indeed, an active Reddit\footnote{See \texttt{http://reddit.com}} or Github\footnote{See \texttt{http://github.com}} account can be seen as a ``personal badge of pride''. Furthermore, political involvement seems to have risen recently. Not-for-profit initiatives such as \textit{Civic Commons}\footnote{See \texttt{http://civiccommons.org/}} ---a group that aims to help governments by achieving efficiency and transparency through IT infrastructures, is one of many examples of this trend.



% 5. Position towards commercialism and capitalism
\subsubsection{Hackers and Capitalism}

Hackers have traditionally held little contempt for capitalism itself. Rather, their disdain is directed towards certain practices that are common (but not necessary) within the system, namely excessive bureaucracy and restriction of access to information. Perhaps as a result of his depiction of hackers' unorganised beginnings, Steven Levy has failed to make this distinction, suggesting ---inaccurately--- a general incompatibility between the original principles of hackerism and capitalist practices. Indeed, hackers have historically dealt with capitalism in a manner not unlike that with which they treat other aspects of their life: by applying creativity and lawful subversion\footnote{I use this oxymoronic term deliberately to emphasise the contradiction many subcultures experience in regards to their political stance and rhetoric as described throughout this chapter.} and achieving unexpected results.

It is only when commercial practices become at odds with The Hacker Ethic that clashes or ideological conflicts between the two worlds arise. Richard Stallman, intellectual leader of the movement and known for his reluctance towards even the slightest compromise, best summed it up by declaring that ``redistributing free software is a good and legitimate activity; if you do it, you might as well make a profit from it'' \citep[p. 65]{stallman02}.

Stallman's statement reveals the \textit{hack} on capitalism, as it applies to software: by allowing freely-distributable programs to openly compete against proprietary ones, hackers have forced software companies to either join the trend, leverage their existing market shares by locking in customers \citep{lee08} or come up with alternative revenue strategies. Ever pragmatists, such is the way hackers exercise defiance: by abiding and subverting creatively within the boundaries of the law: practising hands-on evolution rather than revolution.

Levy's failure to recognise this fact is surprising, given that \textit{Hackers} constantly hints at it. One can read, for instance, how hackers' attitudes towards IBM deeply contrasted those towards another corporation, DEC (Digital Equipment Corporation), for which they held a sense of admiration, perceiving it as being less laden with bureaucracy, more efficient and even worthy of admiration, to the point where some of the original MIT members sought and found paid positions within it.

Needless to say, much has happened since the 1984 release of the book. Its recent 25\textsuperscript{th} anniversary was celebrated with an article in Wired Magazine (that later became an addendum to the new edition) in which Levy re-visited many of his subjects, while also introducing new, younger characters, portrayed as part of an ongoing generational shift. In his article, Levy acknowledges that new hackers have assumed a less bellicose position towards businesses, but still fails to specify the source of the initial opposition ---\emph{some practices} as opposed to the system itself:

\begin{quote}
\dots hacking's values [today] aren't threatened by business ---they have conquered business. Seat-of-the-pants problem-solving. Decentralized decisionmaking. Emphasizing quality of work over quality of wardrobe. These are all hacker ideals, and they have all infiltrated the working world \citep{levy10}.
\end{quote}

The ``hacker ideals'' Levy speaks about have never been in conflict with the ``working world''. Indeed, some of his initial subjects went on to found and lead multi-billion dollar corporations, while many others tried but failed to achieve the same goal ---a quarter of a century earlier. 

This unacknowledged communion, however, has not meant complete assimilation or passive acceptance. In a fashion not unlike that of other movements of the 1960s and early 1970s, hackers have indeed been known for expressing frustration towards peers whom they perceive as having ``sold out'' or become ``corrupted''. Tony Moore argues that countercultural contempt for the establishment has not traditionally been filled with true revolutionary sentiments, since the very existence of well-established institutions is essential as reference parameter from which to rebel against. Without a manifest presence of such institutions there can be no rebelliousness. To hackers, The Man does not necessarily mean The Business or The Government by itself. Instead, it is the \emph{over-bureaucratic} business or the \emph{secretive} government agency. Thus, the ``sell-outs'', in hackers' eyes, are those who associate themselves with institutions seen as contrary to the Hacker Ethic, itself not  immune to the passage of time.

% 6. Nostalgia

\subsubsection{Nostalgia}

As the Hacker Ethic slowly evolves and reshapes itself, so do hackers' group identities. This includes natural and somewhat expected manifestations of denial from the part of older members, who often fail to recognise younger ones as genuine representatives of their group. Inherent in this progression of constant adoption, evolution and denial may be a process of mythification as theorised by \citet{lave91}, who argue that generational shifts amongst communities often tend to spur criticism in the form of nostalgic complaint regarding newcomers' corruption of the original ideals. From this perspective, accusations from the ranks of original or older Hackers towards younger ones in terms of ``selling out'' should come as no surprise and instead be seen as a predictable pattern consistent not only with other coeval groups but historically with almost any collectivity.

By the year 1984, when Steven Levy released \textit{Hackers}, at least three major generational shifts had taken place amongst the ranks of the hackers at MIT and California, while many other groups had begun forming around the world. The unique character of Levy's book lies in the fact that it served both as an initial manifesto and as a way to preserve and spread what had been to that point a mostly oral tradition, thus helping perpetuate the original mythology.

As an epilogue to \textit{Hackers}, Levy documented the story of Richard Stallman, ``the last of the true hackers'', a man so committed to the movement and its ideals that he dedicated his entire life to ---quite literally--- spreading its gospel\footnote{Stallman is known for parodying traditional religious rituals and icons, jokingly dressing up as \textit{Saint iGNUtius} (using a large robe and an old disk drive platter as a halo), whom, he proclaims, is the leader of \textit{The Church of EMACS}. See \texttt{http://stallman.org/saint.html}.}. Stallman grew dissatisfied with what he described as ``the decay of the Hacker Ethic'' \citep[p.415]{levy84}. Where his peers (and elders) moved on from true hackerism to join bureaucratic and tight organisations, he remained (and still remains) unmoved in his convictions and ideals.

Stallman is best known for his perseverance, relentlessness and for authoring the \textit{GNU GPL} ---the widely-used free software license. As a result, he has become both a source of inspiration and aversion (on account of his perceived intransigence and zeal) amongst those close to the movement, in the process achieving the status of a mythical ``founding father'' \citep{jackson98}. While his case shares many characteristics with the processes of mythification from other subcultures, Stallman's is unique in a number of ways. He is, no doubt, the last \emph{active} representative of the original MIT faction that started the hacker subculture, even if his activities these days have more to do with preaching than with hands-on software hacking.

Seen from a wider perspective, Stallman's merit lies in his capacity to convert his own nostalgia into a constant and permanent driving force for activism. In \textit{Hackers} as well as in its new addendum, Levy portrays Stallman as someone filled with nostalgia and angst: ``(his) eyes moistened as he described the decay of the Hacker Ethic'' \citep[p.415]{levy84} and even some suicidal tendencies. He has been quoted to say:

\begin{quote}
I have certainly wished I had killed myself when I was born. In terms of effect on the world, it's very good that I've lived. And so I guess, if I could go back in time and prevent my birth, I wouldn't do it. But I sure wish I hadn't had so much pain. \citep{levy10}
\end{quote}

Through his quote, Stallman also candidly acknowledges his labour as a leader of the movement. Indeed, it is not few would deny that his efforts have been a necessary cause in ensuring the continuity of the lifestyle and principles developed in the early days of the MIT Railroad club. ``What happened to the hackers of yesteryear?'', Levy asked himself. ``Many had gone to work for businesses, implicitly accepting the compromises that such work entailed''\footnote{As I have argued earlier, I believe ``working for businesses'' does not necessarily imply a compromise on the part of hackers.}. Contrastingly, Stallman has never ceased to see himself as the enforcer of a messianic mission to preserve and expand his own interpretation of the Hacker Ethic.
%, as analysed in section \ref{freedom}.


% Place paper conclusions here















%The dilemma which faced the avant-garde already in the 'fifth and sixth decades of the nineteenth century' was that it depended on the capitalist class it hated, the bourgeoisie, for its economic existence: hence the ambivalent motifs of ressentiment and worship which the modern avant-gardes have inherited from Flaubert, Manet, and Baudelaire. --lawson et al in farewell to the avantgarde

%The avant-garde is seen as a procession of ever more opaque and transitory sects, furiously interrogating the parameters of their chosen media and auto-destructing when their naive commitment to the new and the unique becomes appropriated by the Culture Industry and turned into yesterday's commodity, yesterday's advertising slogan


%Terry Eagleton sees postmodernism as a parody of the revolutionary art of the twentieth century avant-garde . . . whose Utopian desire for a fusion of art and social practice is seized, distorted and jeeringly turned back upon them as dystopian reality. Postmodernism . . . mimes the formal resolution of art and social life attempted by the avant-garde, while remorselessly emptying it of its political content; Mayakovsky's poetry readings in the factory yard become Warhol' shoes and soup-cans.8












%I also found a third group: the present-day heirs to the hacker legacy, who grew up in a world where commerce and hacking were never seen as opposing values levy2010

%A new generation of hackers has emerged, techies who see business not as an enemy but as the means for their ideas and innovations to find the broadest audience possible levy2010




%by attributing to both the commodities they made and their symbolic consumption the aura of autonomous art and déclassé origin, bohemians symbolically performed the status of social outsider and the natural aristocrat superior to the business faction of the bourgeoisie moore p.17

%For example in the 1950s the official communist critique of bohemia was stated by Komsomolskaya Pravda in an attack on the ‘beatnik’ craze, that asked how these ‘coffee cup anarchists’ can ‘protest against the ruling class when they themselves are members of it?’ moore p.19

% While wark separates Hackers (one class) from vectorialists (another class) much like marx does, Stallman puts them on the same level, and argues that things need to change:
% "In the short run, this is true. However, there are plenty of ways that programmers could make a living without selling the right to use a program. This way is customary now because it brings programmers and businessmen the most money, not because it is the only way to make a living" ---gnu manifesto

% ars gratia artis, hacks gratia 
% bohemian habits, adapt to Hackers
% ethics, manifestos, etc
% rejection of the establishment
% connotation of criminality (t.moore)
% nostalgia. (the last of the great Hackers)
% moore calls bohemianism a ``collective strategy'' that, by denying commercialism, does exactly the opposite

% difference: socially 'challenged' individuals with very little social ties outside their own 'bubble'
% bohemians:anti-bourgoisie, Hackers ... hhhmm?
% internet replaces the 'smokey pubs' of socialisation for wannabes and newbies, however, Hacker CONs have a lot of drinking and other leisurely activities
% I would like to see GNU development supported by gifts from many manufacturers and users, reducing the cost to each [stallman, academic culture and gift economy]




%t,.moore calls bohemians 'radical innovators', which leads us to.... section 'Innovators and researchers'


\subsection{Freedom of Information, Freedom of Software}
\label{freedom}

%A characteristic that has made the Hacker stance more interesting is the fact that they have found ways to skilfully apply the system's own rules against it. In other words, they have figured out a way to \emph{hack it}. The methods through which they exercise this are examined in section \ref{freedom}. 


The hacker subculture's country of origin has deep implications when it comes to their views on freedom of information. Prior to the Copyright Act of 1976, the United States held an undeclared tradition of disdain towards restrictions on the use of information. Its democratic foundations have, for centuries, equated access to information to accountability and good government. Furthermore, as a result of its revolutionary origins, the government explicitly encouraged copyright violations for works coming from abroad. According to \citet{khan06}, the country's first copyright act declared that `nothing in this act shall be construed to extend to prohibit the importation or vending, reprinting or publishing within the United States, of any map, chart, book or books \ldots by any person not a citizen of the United States'.

% Gates puts the argument in perspective by pointing out that centuries ago, European publishers printed American writers’ works without compensation. “Benjamin Franklin was so ripped off — he could have written exactly what I wrote in that letter,” he says. levy2010

Today, the situation is clearly quite different. The United States went from being a net importer of cultural works to being perhaps their largest exporter. This, combined with the decline of manufacturing in developed nations has led to increasing zeal over the control of cultural material, regardless of the format, resulting in increasing restrictions in terms of its distribution and flow, and achieving an effect of artificial scarcity and commodification. Ideas, along with cultural works, are perceived as sources of tangible monetary value and competitive advantages that are to be carefully controlled and prudently rationed.

The implications of cultural commodification beyond their effects on hackers are perhaps beyond the scope of this work, yet increasing restrictions applied to cultural and creative works have undoubtedly had tangible consequences on society as a whole.

For hackers, however, the free flow of information constitutes a necessary cause. In his book, \textit{A Hacker Manifesto}, McKenzie Wark \citeyearpar{wark04} describes this situation as the motive force for this century's class struggle, in his opinion being waged by ``manufacturers'' of cultural works ---hackers--- and the beneficiaries of their production (or rather, commercialisation), referred to as vectorialists. On cultural commodification, \citeauthor{wark04} argues that `commodified life dispossess the worker of the information traditionally passed on outside the realm of private property as culture' \citep[v. 28]{wark04}, drawing clear parallels to Marx's workers' struggle, yet not about means of production, but rather, `freeing information from its material constraints' \citep[v.4]{wark04}.

%Big business may stumble upon and commodify their breakthroughs, but hackers will simply move on to unexplored frontiers. “It’s like that line in Last Tango in Paris,” O’Reilly says, “where Marlon Brando says, ‘It’s over, and then it begins again.’” levy2010

Wark's analysis ---while extremely enlightening--- falls short on account of its abstraction. To fully understand the issue of freedom of information in the context of the origins of hackerism, one finds more concrete precedents by turning, again, to Steven Levy's book.  As Levy \citep[p.5]{levy84} recalls, the movement itself was born out of contempt for the ``priest-like'' figures who sought to monopolise and restrict the use of computer resources\footnote{Use of the word ``priest'' draws an implicit parallel between ancient means of control through literacy and today, through The Computer.}. As outsiders, the very first Hackers were denied the possibility of using and studying the powerful new machines, having to resort to clandestine tactics to sneak inside the buildings late at night to gain access to the room-sized contraptions. 

As it flourished, hackerism embraced the mantra of free-flowing information, building its own young subculture on top of it. The Artificial Intelligence Laboratory at MIT became the unofficial headquarters of the group, a place where constant cooperative competition, respectful acknowledgement of others' achievements and a mentality of studying and building upon what existed became a non-negotiable norm.

Yet, as personal computers became more and more popular, the trend began to be reversed. By the time Steward Brand famously coined the phrase \textit{Information Wants to be Free} in 1984 \citep{wagner03}, the computer revolution was in full swing. Apple had led the massification of personal computers and was already a multi-million-dollar company, while IBM and others attempted to catch up with varying degrees of success. As a result, software became a priced asset ---one that had to be guarded by means of withholding its source.

Pressure to preserve increasingly large amounts of software as trade secrets reached even the sacred confines of the Artificial Intelligence Lab, where many, headed by the relentless Stallman, refused to give in to the trend, becoming, in the process, some of its most vocal critics. Stallman's \citeyear{stallman85} \textit{GNU Manifesto} became not only a declaration of principles:

\begin{quote}
I consider that the golden rule requires that if I like a program I must share it with other people who like it. \dots I refuse to break solidarity with other users in this way. I cannot in good conscience sign a non-disclosure agreement or a software license agreement.
\end{quote}

\noindent
But also a call to arms amongst hackers to fight those restrictions the way they knew how to:

\begin{quote}
GNU, which stands for Gnu's Not Unix, is the name for the complete Unix-compatible software system which I am writing so that I can give it away free to everyone who can use it. Several other volunteers are helping me. \ldots So that I can continue to use computers without dishonor, I have decided to put together a sufficient body of free software so that I will be able to get along without any software that is not free.
\end{quote}

The document described his vision of a world where sharing information was to be considered an act of neighbourly kindness rather than a crime. It also presented a detailed outline of his strategy, which began with his pledge to write GNU, a free replacement to Unix\footnote{Unix was, back then, the prevailing operating system. With the advent of GNU/Linux and Mac OS X, Unix, or rather Unix-like systems, have become immensely relevant once again.}, from the ground up ---a monumental task--- along with his resolution to resign from the Artificial Intelligence Laboratory at MIT in order to avoid `any legal excuse to prevent me from giving GNU away' \citep{stallman85}.

While the essence of the GNU Manifesto remains current to this day, Stallman's philosophical and strategic plans have grown and matured. Today, he is considered a pioneer and the father of \textit{free software}\footnote{Whilst, in practice, \textit{free software} and \textit{open-source software} may be used interchangeably, the former is usually associated with those whose concerns are more philosophical, like Stallman, while the latter is used by advocates who see more practical benefits to its use.}, not only due to his initiative to write GNU but also, to a great degree, for having authored the immensely popular \textit{GPL}, or GNU General Public License, the first \textit{copyleft} software license. Known for being a clever \textit{hack} on copyright law, the GPL uses the restrictions copyright grants on works such as software to ensure that its terms, which mandate any derivative works to remain open, are upheld and respected.

Within the \textit{Jargon File} \citep[p.334]{raymond93}, the case for information openness is argued in the entry that references the Hacker Ethic: `it is an ethical duty of hackers to share their expertise by writing open-source and facilitating access to information and to computing resources wherever possible'. While the presence of the term \textit{open-source} can lead the reader to conclude that this was a rather late addition to the file\footnote{The term `open source' was coined in 1998. See \citet{osi12}}, it serves as evidence to the fact that the ethos has remained unchanged to this day, and throughout the history of the subculture, despite claims, such as Levy's own recent commentary to his own text, that hacker values no longer centre around freedom of information but solely on the desire for tinkering and exploring \citep{levy10}.

As mentioned earlier, the GPL is perhaps the first example of this trend, which can be considered one of the pillars of the movement known today as \textit{copyleft}. As a system for countering what is perceived to be an excessively and increasingly restrictive copyright law, copyleft confronts these restrictions \emph{from within}, by imposing limits to \emph{lack} of openness, disclosure and sharing to works that are licensed by any one of the many flavours of licenses available to choose from.

Young hackers who today remain faithful to the essence of the Hacker Ethic regard never-ending pressure for increasing restrictions on the flow of information much with the same eyes, something to be not only opposed to and protested but also, ``subverted'' and ``transgressed''.



% restrictions reduce the amount and the ways that the program can be used. This reduces the amount of wealth that humanity derives from the program. When there is a deliberate choice to restrict, the harmful consequences are deliberate destruction.
% Since I do not like the consequences that result if everyone hoards information, I am required to consider it wrong for one to do so.


%``To hack is to produce or apply the abstract to information and express the possibility of new worlds.'' wark04 1.14
% 'The class interest of Hackers lies in freeing information from its material constraints' wark04 
% making the state the monopolist of property has only produced a new ruling class, and a new and more brutal class struggle -wark24
% Commodified life dispossess the worker of the information traditionally passed on outside the realm of private property as culture -wark28
% Hacking as a pure, free experimental activity must be free from any constraint that is not self imposed. Only out of its liberty will it produce the means of producing a surplus of liberty and liberty as a surplus. wark 197
% The gift of information need not give rise to conflict over information as property, for information need not suffer the artifice of scarcity once freed from commodification. wark 204
%Education is slavery, it enchains the mind and makes it a resource for class power. wark 48

\subsection{Hackers and Education: It's Complicated}

By virtue of their very origins, hackers have an interestingly contradictory stance when it comes to formal education. As has been argued earlier, their very genesis is the product of a University environment, yet it came to be as a defiance of traditional structures. Brian Harvey \citeyearpar{harvey86} elaborates on the etymology of the word, as it originated and evolved in the 1950s and 60s, whilst pointing out an interesting contrast:

\begin{quote}
Popular opinion at MIT posited that there are two kinds of students, tools and hackers. A ``tool'' is someone who attends class regularly, is always to be found in the library when no class is meeting, and gets straight As. A ``hacker'' is the opposite: someone who never goes to class, who in fact sleeps all day, and who spends the night pursuing \emph{recreational activities} rather than studying. (emphasis added)
\end{quote}

It is the very nature of hackers' recreational activities that leads to their learning experiences. Hackers will not follow the path of least resistance, which, in their case, represents the traditional go-to-class, do-your-home\-work mantra of the classic academic model, but will, in contrast, be mostly driven by large amounts of intellectual curiosity, which in turn is a source of pleasure. This abundant source of intrinsic motivation creates a number of both benefits and challenges.

Benefits, whilst seemingly clear, deserve mention. Hungarian psychologist Mihaly Csikszentmihalyi famously correlated enjoyment of an activity to increased concentration and ability through what he described as a state of ``flow'' \citep{csikszentmihalyi75}. Flow is achieved when an individual's skills are appropriate for the task at hand, but are met with a certain amount of challenge, itself a pathway to discovery and learning. Flow, thus, requires the use of individuals' skills to the point where their activities are performed ``naturally'', without conscious inner reflection, resulting in increased performance and faster learning curves. Hackers have noticed this phenomenon, in their own way, and named it accordingly. The Jargon File elaborates:

\begin{quote}
\textbf{Hack mode}: a Zen−like state of total focus on The Problem that may be achieved when one is hacking (this is why every good hacker is part mystic). Ability to enter such concentration at will correlates strongly with wizardliness \citep[p.331]{raymond93}
\end{quote}


\citet{lakhani03} have further proposed that curiosity-driven hackers can also become more creative. Basing their argument on the work of \citet{amabile96}, they suggest curiosity leads to \textit{heuristic} approaches to problem solving in hackers, in opposition to otherwise \textit{algorithmic} resolutions. As an example, the authors point to the development of a printer driver ---a task not easily perceivable as \emph{creative} by someone not connected to the project. Yet, anyone who is directly involved with the challenge can appreciate the creative effort that goes into it. \citet[pp.13--33]{levy84} argues that hackers apply most of their creativity in the process known as ``program bumming''\footnote{To ``bum'' a program is to reduce it to the absolute fewest number of lines without affecting its outcome.}. Indeed, he goes as as far as describing program bumming as ``artful'', in concordance to principle five of \textit{The Hacker Ethic}, ``You can create art and beauty in a computer''.

In spite of these benefits, however, aversion for established curricula comes at a price. Hackers' engagement in a project or activity can quickly fade if they find something more interesting to devote their attention to. Whilst, as discussed, their focus and concentration may be abundant, their commitment may not necessarily match the trend. In his landmark essay \textit{The Cathedral and the Bazaar}, \citet{raymond99a}, documented several instances of Free/Open Source software that were, for one reason or another, abandoned by their founders. Indeed, one of the guidelines the author suggests for writing good software using the \emph{Bazaar} method\footnote{Raymond contrasts the traditional top-down approach to building software, common back then in corporate environment and likened to the building of a large cathedral, to the then-revolutionary bottom-up method devised by Linus Torvalds for the Linux kernel, seemingly disorganised and mostly de-centralised, likened by Raymond to a ``great babbling bazaar of differing agendas and approaches'' \citep[p.3]{raymond99a}.} states that ``When you lose interest in a program, your last duty to it is to hand it off to a competent successor'' \citep[p.6]{raymond99a}.

\citeauthor{raymond99a} recognised the problem of loss of commitment and sought to propose a suitable solution to constant abandonment of what constituted, even in 1999, a vast repository of free software. He did not attempt to change developers' minds about such abandonment, as, while not making this explicit, he likely recognised this as an inherent hacker characteristic. In light of this fact, every hacker project ---be it hardware produced out of a hackerspace or a distributed software effort--- needs to find suitable strategies for torch-handing. This may be done by organising contingencies around the inherently meritocratic nature of their relationships.

Indeed, principle number 4 of \textit{The Hacker Ethic} states that ``Hackers should be judged by their hacking, not bogus criteria such
as degrees, age, race or position''. As Levy himself has suggested, this trait stems not from a desire to do the right thing, but is, in fact, a rather pragmatic approach on their part. By welcoming and embracing those who are good at their craft, hackers satisfy their general state of intellectual curiosity and advance their projects forward. This will, in most cases, result in a red tape-free, intellectually stimulating environment. Furthermore, this practice goes hand in hand with peer-based group learning, a process which has greatly evolved, alongside technology.

With the emergence of the world wide web and prior to the popularisation of hacker conventions, simply known as \textit{cons} and the dawn of hackerspaces, sharing and learning processes began taking place online. Hackers were no longer restricted to working with and acquiring knowledge from peers who were physically close. This led to an explosion of information flow and the increasing complexity of software projects, which ultimately resulted in the creation of the Linux kernel, among many others. This shift had obvious consequences in how hackers learned and cooperated. They became more isolated in the physical sense and their interactions became mediated by networked publics. In detailing how this relates to their education, \citet[p.75]{himanen01} called this phenomenon ``The Net Academy'', in reference to Plato's Academy. Using this term, \citeauthor{himanen01} envisaged an educational model, that is, a model for higher education, based even more closely on hacker principles: more horizontal, more open and less formally structured, thus allowing pupils to follow their own interests.

If one is to judge by results yielded, the model is, indeed, highly successful ---at least with regards to the production of software. The GNU/Li\-nux operating system has gone from being a fringe project to powering most of the web's infrastructure, not to mention a wide array of devices such as mobile phones\footnote{Google's Android operating system, which represented, as of September, 2011, 56\% of the mobile market \citep{jackson12}, is powered by GNU and Linux.}, home appliances and even spaceships \citep{debian97}. Furthermore, many of today's large corporations (Apple, IBM, Google, Facebook) as well as up and coming start-ups all benefit from and contribute to the Free/Open Source software environment.

It remains to be seen, however, if these methods for learning and making are equally successful under the newer circumstances created by the emergence of hackerspaces. The historical evidence seen throughout the dawn and flourishing of the subculture in the times pre-dating the web certainly seem to suggest so, yet the challenges and opportunities of this new phenomenon deserve their own analysis, the first step of which is to understand how hackerspaces came to be.


% Points I want to make:
% Hackers do not sit normal class
	% Hackers do not take the "path of least resistance"
	% Hackers are driven by curiosity so lack of motivation is not an issue
	% Hackers will deviate if they find something more interesting

	% Hackers are a meritocracy
% However, they have their own group dynamics for learning
	% which involves little control
	% yet they will take money
% Ultimately this money yields results 



%most represent a more intellectually curious, highly principled, authority-averse, scientifically open-minded attitude conti05
% The hackers are largely self- and peer-taught. But do not assume that just because no computer sci- ence diplomas hang on their walls (though many do) they are not worth engaging conti05





% To produce is to repeat; to hack, to differentiate wark 160

%The Hacker class have an ambivalent relationship to education. The Hacker class desires knowledge, not education. The Hacker comes into being though the pure liberty of knowledge in and of itself -wark 55
% Hacker knowledge implies, in its practice, a politics of free information, free learning, the gift of the result to a network of peers wark55


%Still, the TX-0 was the center of his college career, and he shared the common hacker experience of seeing his grades suffer from missed classes. It didn’t bother him much, because he knew that his real education was occurring in Room 240 of Building 26, behind the Tixo console. Years later he would describe himself and the others as “an elite group. Other people were off studying, spending their days up on four-floor buildings making obnoxious vapors or off in the physics lab throwing particles at things or whatever it is they do. And we were simply not paying attention to what other folks were doing because we had no interest in it. They were studying what they were studying and we were studying what we were studying. And the fact that much of it was not on the officially approved curriculum was by and large immaterial.” levy p.23

%Computer programming must be a \emph{hobby}, something done for fun, not out of a sense of duty or for the money. (It's okay to make money, but that can't be the reason for hacking)

%Node:hack mode, Next:[6465]hack on, Previous:[6466]hack attack, Up:[6467]= H = hack mode n. 1. What one is in when hacking, of course.2. More specifically, a Zen−like state of total focus on The Problem that may be achieved when one is hacking (this is why every good hacker is part mystic). Ability to enter such concentration at will correlates strongly with wizardliness; it is one of the most important skills learned during larval stage. Sometimes amplified as `deep hack mode'. Being yanked out of hack mode (see [6469]priority interrupt) may be experienced as a physical shock, and the sensation of being in hack mode is more than a little habituating. The intensity of this experience is probably by itself sufficient explanation for the existence of hackers, and explains why many resist being promoted out of positions where they can code. See also [6470]cyberspace (sense 2). Some aspects of hacker etiquette will appear quite odd to an observer unaware of the high value placed on hack mode. For example, if someone appears at your door, it is perfectly okay to hold up a hand (without turning one's eyes away from the screen) to avoid being interrupted. One may read, type, and interact with the computer for quite some time before further acknowledging the other's presence (of course, he or she is reciprocally free to leave without a word). jargon file p.


\subsection{Hackerspaces}




\begin{epigraphs}
\qitem{\textit{The virtual is the true domain of the Hacker. It is from the virtual that the Hacker produces ever-new expressions of the actual. To the Hacker, what is represented as being real is always partial, limited, perhaps even false. (\ldots) To hack is to release the virtual into the actual, to express the difference of the real}}{McKenzie Wark \citeyearpar[v.74]{wark04}}

\qitem{\textit{Indeed, serious hackers primarily exist as hackers on-line}}{Manuel Castells \citeyearpar{castells01}}
\end{epigraphs}

Statements not unlike the two epigraphs above are common in academic literature from the first half of the 2000s. It is important, thus, to understand how and why hackerspaces have thrived when physical presence in the context of co-operative commons-based collaboration was widely declared irrelevant.

While, in their most recent incarnation, hackerspaces are a relatively recent phenomenon, there is ample precedent for them. Those who form them see their spaces as a natural evolutionary step stemming from hippie communes. Hacker and artist Johannes Grenzfurthner argues that hackerspaces (along with squat houses, alternative cafes, farming cooperatives, etc.) ``established a tight network for an alternative lifestyle within the heart of bourgeois darkness'' \citep{grenzfurthner09}.

Nick Farr sees today's hackerspaces as a ``third wave'' in a reference to the work of Alvin Toffler, referencing early precursor spaces in both North America and Europe as first and second waves respectively. Whilst Farr's analysis is fascinating, I choose to view the origin of hackerspaces from a different perspective and argue that they are the product of two distinct sets of circumstances: a reconsideration of the death of distance paradigm and the increasingly low cost of electronic hardware.

The latest wave of hackerspaces (what Farr calls the ``third wave'') came to be as a result of very specific circumstances. In 2007, a group of 35 North American hackers (led by Farr himself) embarked on a trip to Europe in what came to be known as \textit{hackers on a plane}. Their purpose was to physically attend a conference, the CCC, or \textit{Chaos Communications Camp}, which was to be held in Germany and organised by the famed \textit{Chaos Computer Club}. Upon their return, several of these hackers decided to apply what they witnessed and learnt while in Germany and founded some of the most iconic spaces in the United States: Mitch Altman started Noisebridge in the bay area, Bre Pettis began NYC Resistor in New York and Farr founded HacDC in Washington DC. From there, the trend began to quickly spread across the world.

The \textit{Chaos Computer Club} itself is considered the European precursor to modern day hackerspaces. Before its foundation in 1981, earlier groups of young computer enthusiasts had formed in the United States, most notably MIT's Tech Model Railroad Club and the California-based Homebrew Computer Club. Yet, of all three, not only is the CCC the only surviving club, it also shares a direct ``evolutionary'' link to hackerspaces \citep[p.84--86]{pettis11}.

Wau Holland founded the CCC in West Berlin in 1981 together with a group of people who took an interest in technological advances, particularly those related to computers. During those days Germans saw the computer revolution as means to ``bring about more surveillance and fascism'' \citep[p.84]{pettis11}, yet the hackers of the CCC became interested in the machine's liberating potential. Prior to the spread of the Internet, their attention focused on ``opening'' tightly-regulated and monopolised early network communications through custom designed and built modems, using plumbing pipes for coupling telephone headsets.

These aptly-named ``dataloos'' then served as devices through which Germans connected their home computers. Yet, even as networks grew in size, the CCC always remained a tight group of individuals whose main connection was made possible by the bonds of mutual co-presence. The rise of the web did not hinder this trend; rather, it strengthened it, as local groups began to spawn across Germany and neighbouring countries. The trend was reinforced by the institution of \textit{Chaos Communications Congresses} (indoors) and \textit{Chaos Communications Camps} (outdoors). Like the clubs themselves, these events greatly benefited from the dawn of the web:

\begin{quote}
The dot-com boom was ramping up and CCC grew from about 250 people to 1500. There is a regional group in every city and because the first meeting of the CCC happened on a Tuesday, all the groups meet weekly on Tuesdays. While Tuesday CCC meetings are for members only, many regional branches have a public night for talk and discussion either weekly or monthly on a Thursday \citep[p.84]{pettis11}.
\end{quote}

As direct descendants of the CCC, today's hackerspaces are organised around the exact same ethos. To members, the convenience of the web makes it not a means of isolation but a conduit for facilitating assembly, which itself fosters an environment ideally suited for experimentation, not only with software but physical objects as well. Indeed, \citepos{moilanen10} survey of hackerspaces confirms that ``hardware development and hacking'' is the single most common activity within these spaces. The trend clearly contradicts Steven Levy's statements, proving that interest in hardware hacking is on the rise, a trend that had not been observed since the days of the Homebrew computer club. This resurgent interest comes hand in hand with the boom of the \textit{Maker} movement and \textit{open-source hardware}.

Exponential improvements in computing performance, in line with \textit{Moore's Law}\footnote{Moore's Law is a trend whereby computing power is commercially duplicated every two years. It is named after Gordon E. Moore, who first predicted it in 1965.} have had tremendous impact in the declining general cost of electronic components. This fact, combined with wide availability of open schematics has led to extraordinary interest in electronics and the phenomenon been dubbed as the \textit{Internet of things}, a network of interconnected physical, digital devices that ``talk'' to each other in a similar way as computers do.

At the heart of this interest is the \textit{Arduino}, a credit card-sized micro-controller that serves as a logic unit for an endless a of devices, most of which are thought of and built inside hackerspaces. Arduino was developed in Ivrea, Italy, itself derived from Colombian artist and developer Hernando Barrag\'{a}n's \textit{Wiring} project \citep{reas10}.

A recent article on the subject by \textit{The Economist} \citeyearpar{economist12} outlines some of the more interesting (or rather, quirkiest) Arduino-driven inventions:

\begin{quote}
\ldots plants that send Twitter messages when they need watering, a harp made of lasers, an etch-a-sketch clock, a microphone that serves as a breathalyser, or a vest that displays your speed when riding a bike
\end{quote}

As the article points out, the ``Arduino revolution'' is a product of its low cost (\$20 USD for a basic board) and a sizeable and increasing array of accessories (touch screens, microphones, sensors), made possible by the system's open schematics, which, rather than hindering its potential, has led to hundreds of thousands of units sold \citep{economist12} while also paving the way for the development of a number of clone boards, all of which share the same de-facto standards and work with the same basic programming language, called \textit{Processing}.

Arduino is also at the heart of 3D printers: machines that produce physical objects out of digital models by means of extruding hot plastic through a nozzle in a controlled and systematic way. As a general rule, most (if not all) hackerspaces begin their life building a 3D printer, not only because it is a key project for the purposes of acquiring essential skills, but also because, once assembled, a 3D printer is a crucial replicator, allowing hackers to build parts for other projects in an easy, convenient and inexpensive way\footnote{The implications behind the fascinating world of 3D printers are analysed in chapter 'making'}. In briefly referring to 3D printers and hackerspaces, the article in \textit{The Economist} says:

\begin{quote}
Many [hackerspaces] are organised like artists’ collectives. At Noisebridge, a hacker space in San Francisco, even non-members can come and tinker ---as long as they comply with the group’s main rule: to be “excellent” to each other. ``\emph{The Internet is no substitute for a real community},'' says Mitch Altman, a co-founder of Noisebridge. (emphasis added).
\end{quote}

The quote provides a simple, yet enlightening perspective. In enthusiastically adopting open-source hardware and leveraging the benefits of online communications hackers and hackerspaces have come full circle as descendants of earlier collectives, re-discovering and embracing the benefits of co-presence for their own, unique purposes. In doing so, they have enabled themselves to work with physical objects, aided by dropping hardware costs, bringing to an end an era in which the ideal ---indeed the only--- way to hack, was through the authorship of aeriform software.


% -----------------------
% MINI CONCLUSSIONS
% -----------------------


\vspace{30pt}
\centerline{$\diamondsuit$}
\vspace{30pt}



% conclusion: hackers are more pragmatic bohemians
%\subsection{Hackers: Pragmatic Idealists}

The same social circumstances that led to an explosion of countercultural movements in the decades after the post-war were decisive in shaping the emergence of the hacker subculture, one that was born inside the confines of the Massachusetts Institute of Technology but quickly found its way to the rest of the United States and elsewhere by virtue of the invention of the microprocessor. Although revolving around technologies, hackers were part of the same ``salad bowl'' environment that fostered the genesis of dozens of movements: hippies, free speech and civil rights activists, feminists and many others who, together, were seen as a force swaying away from social traditional values and established world-views, and whose main unifying characteristic was an explicit desire to distance themselves from those views and values, which they perceived as corrupted. Despite their shared origins and common ground, however, hackers followed a more pragmatic approach in their path to self-discovery, preserving what they saw as useful and making conscious efforts to change things by means of subtle and unexpected ---but usually lawful--- manipulation of the rules of the environments that concerned them. The hacker mythology and its ethos are based on the expansion of freedom of speech towards cultural manifestations and information with an emphasis on software ---the language of computers.

For the hacker subculture, freedom of information became a primary ideal, as it was thought that only through it would personal freedom be fully achieved: \textit{information shall set you free}. Furthermore, thirst for information, manifested in inherent curiosity has provided ideal circumstances for the subculture to flourish, both within the confines of networked publics and, increasingly, through physical co-presence, as a result of a number of things, including a rise in their numbers, dropping hardware costs and, somewhat ironically, the use of online tools, which, rather than promoting isolation, has allowed them to communicate and organise efficiently.

This chapter concludes the ``lead-in'' of this dissertation. The core, starting with chapter [4], ['making'], aims to analyse the consequences of these developments, in order to better understand hacker sociality, skills transmission and potential for innovation.



%It is worth noting that, however strong the above quotations may sound, they do not suggest that hackers are loners by nature. Their sociality, however, takes place within well-defined boundaries comprised of like-minded peers and mostly ignored by outsiders.

%Indeed, the emergence of Hackerspaces is solid proof the high degree of social activity amongst these groups. Recently, \citet{coleman10} studied several aspects of hacker sociality, through the examination of conventions or \textit{cons}. Coleman quickly debunked the myth of the lone hacker, arguing instead that they exercise their sociality within their relatively tight circles. The itinerant cons (and recently the more permanent Hackerspaces) increasingly relate to hackers much in the same way in which the Caf\'{e} (or the ``smokey pub'' in Australia) did to bohemians, to whom such spaces were indispensable for their group identity \citep[p.56]{moore07}.


% Unlike the original hackers, Zuckerberg’s generation didn’t have to start from scratch to get control of their machines. “I never wanted to take apart my computer,” he says. As a budding hacker in the late ’90s, Zuckerberg tinkered with the higher-level languages, allowing him to concentrate on systems rather than machines. levy2010