\section{Introduction}
\label{intro}

The rise and mass adoption of the Internet as a multimedia tool for communications has allowed humanity to socialise and collaborate in novel ways that would have been unimaginable just a few years ago. Less than a decade after the Internet ceased to be the exclusive domain of University members and a handful of corporations, people from around the world began to communicate in ways that were, before that, seen only as the stuff of science fiction: video conferencing, three-dimensional environments and media convergence enabled millions to interact and come together in creative and often unforeseen ways \citep{blackman98,syvertsen03,dowding01}.

Groups of like-minded individuals, until then separated by the tyranny of physical distance, almost spontaneously realised that they could now come together to design, author, produce, compile and share wonderful things: computer software, works of art and even, to paraphrase Wikipedia, ``the sum of all human knowledge'' ---all done remotely from the comfort of everyday people's homes and offices. 

Unsurprisingly, interest in these groups grew considerably. Academic journals of varied disciplines began publishing increasing numbers of works on the topic. From a sociological perspective, much of the interest focused on the rise of a new type of ``Habermasian'' Public Sphere \citep{dahlberg01,poster01,gimmler01}, arguing that the Internet had become (or had the potential to become) a public forum for exchanging ideas, discussing and debating, yet one that required no physical presence. Many scholars (myself included) jumped on this bandwagon, praising \textit{Networked Publics} as well-fit, if not ideal, means (places) for discussion and collaboration \citep{boyd08,ito08,moreira09,angel09}.

In the wake of this revolution, it seemed like physical presence began to be seen as a luxury: it was nice to have, but far from a necessity. Yet there were still numerous voices critical of some of the more triumphant claims of the death of distance. Some argued that, while the Internet presented interesting opportunities for enhancing and complementing physical interaction, it could not completely replace it \citep{dahlberg01,dahlberg01b,dahlgren05}. Others strongly maintained that physical proximity was a key factor in technological development and skills transmission \citep{howells00,oinas00,morgan04}. Having been a strong supporter of mediated learning environments, I began to reconsider my position after hearing about Hackerspaces.

I first came across the term \textit{Hackerspace} in 2009, while browsing BoingBoing\footnote{See \texttt{http://www.boingboing.net}}, an online magazine and blog describing itself as ``a directory of wonderful things'' where a post, written by one Mitch Altman, read:

\begin{quote}
If you’ve never been to a hacker conference or a hacker space [sic], you may wonder what a bunch of hackers would do when they get together. Hackers are a very large group of individuals all around the planet who love learning about technology, making it better, and sharing it with the world (\ldots) Hacker spaces are popping up all over the world. These past 12 months have seen so many renting their own space: Philadelphia, New York City, Kansas City, Toronto, San Francisco, Montreal, DC, Vancouver, Paris, Boston, Providence, Chicago (\ldots) [There are] well over 100 spaces on planet Earth where people can get together and share, learn, and work on the next cool thing \citep{johnson09}.
\end{quote}

I immediately became curious. These were the same people about whom I was writing for my Master's Thesis: Free/Open Source programmers, hobbyists and enthusiasts. Right before my eyes, they were gathering again, \emph{away} from Networked Publics and back into the real world. But why? Physical co-presence in a common shared space surely presented logistical challenges. First, it required funding, which mostly appeared to come out of members' own pockets. Second, the task of setting up communal spaces had to require a high degree of coordination, management and a significant enough number of interested people. Yet, against what I considered to be tough odds, these spaces were emerging all over the world, at a rate that was hard to keep track of \footnote{Altman's 2009 estimation of 100 hackerspaces has indeed become obsolete. Hackerspaces.org lists 394 active spaces across the world and many more in ``planned'' status.}.

There is no single or all-encompassing definition as to what a hackerspace is. Hackerspaces.org, perhaps the most comprehensive online resource, defines them as ``commu\-nity-operated physical places, where people can meet and work on their projects'' \citep{hackerspaces11}. Indeed, hackerspaces are diverse and eclectic places. Mitch Altman ---who turned out to be an old-school hacker and hackerspace pioneer, founder of the Noisebridge hackerspace in San Francisco commented: ``It's not easy to say what a hackerspace is, exactly. \emph{You know it when you're in one}'' \citep{altman11}.

Wired magazine rightfully compares hackerspaces to the artist collectives of the 1960s and 1970s, ``located in rented studios, lofts or semi-com\-mer\-cial spaces, hacker spaces (sic) tend to be loosely organized, governed by consensus, and infused with an almost utopian spirit of cooperation and sharing \ldots almost a fight club for nerds'' \citep{tweney09}. In essence, hackerspaces are communities of relatively young people ---30 years is the mean \citep{moilanen10}, skilled in several activities, mostly having to do with electronics and computer technology (programming languages, hardware hacking, soldering) who come together, under one roof, to \emph{socialise}, \emph{learn} and \emph{make things}.

\subsection{Ecosystems for Innovation:\newline{}Relevance of the \emph{Hackerspace} Phenomenon}

\epigraph{\textit{I don't need to remind you of the essence of competition. It's always been quite simple. Any kid working in a garage anywhere in the world with a good idea can put us out of business.}}{Gary Winston. From the movie \textit{Antitrust}}

In determining whether the phenomenon of hackerspaces is worthy of academic study or simply a passing fad, one has to consider its relevance within a set context as well as its current dimensions. Contextually, hackerspaces present a significant opportunity to understand people's needs for physical contact in a world where such contact is no longer essential for a growing number of activities, to the point of causing the term \textit{de-centralisation} to become somewhat of an academic buzzword. Studying hackerspaces will allow for an interesting assessment of the efficacy of online learning in informal environments and how it fares against the traditional physical learning experience while providing insight on hackers' motivations and their need for face-to-face interaction.

Furthermore, their growth rate and presence in all continents certainly suggests that they transcend local cultures and economic boundaries. While still mostly a western phenomenon, hackerspaces have recently emerged in locations as culturally and economically diverse as Yemen, China, Nepal, Mexico and Per\'{u}, amongst many others. Interestingly, and despite their wide international expansion, scholarly work involving hackerspaces has been relatively slow to catch up.

In considering hackerspaces' pertinence as study-worthy communities entirely from a sociological perspective, one may overlook their potential for breeding valuable research and technology. It is not difficult to see apparent similarities between hackerspaces and the mythical garages that have come to symbolise the rise of several hi-tech giants. While some writers argue that ``garage stories'' are nothing more than foundational myths, albeit ones that are encouraged as ``pedagogical tools to train and inspire the young'' \citep[p.239]{kenney00}, the tale of genesis in a garage (or dorm room or as a grad project) does suggest increasingly low barriers of entry to an industry where multi-billion dollar corporations can rise in just a few years \citep{bahrami00}. Indeed, several hackerspaces across the world are involved in projects with real scientific and economic potential: Melbourne hackers are backing a serious effort to land a privately-funded rover on the moon \citep{connectedcommunity11} while others have invented and marketed several successful devices such as the TV-be-gone \citep{bodzin04} and the MakerBot 3D printer \citep{ginn11}.

I share \citepos{kalish10} view that hackerspaces are community-led ecosystems for learning, research and innovation. Whilst the current body of academic work directly dealing with hackerspaces is scarce, interest is growing fast. As detailed in section [enter refernence], I have come across several papers and study proposals in academic literature. Media interest also seems to be growing swiftly in the form of magazine articles \citep{tweney09,dougherty10}, mainstream television news \citep{ginn11} and a documentary currently in production \citep{bunker11}. Furthermore, the recent avalanche of attention garnered by \textit{Wikileaks}, has directed attention towards the organisation's hacker origins and the concept of \textit{hacktivism}. Indeed, hackerspaces are at a stage where they \emph{beg} to be the subject of serious and profound academic enquiry.


%Sociologically, the case for ** a doctoral thesis to the phenomenon of hackerspaces ....
%From the perspective of the sociological sub-field of Science and Technology Studies (S\&TS) ...

\subsection{Of Hackers and Hackerspaces}

Long before the media machineries began associating the word \textit{hacker} with high-tech crime, various groups of well-intentioned, smart and motivated individuals describing each other as hackers played a pivotal role imagining and spawning what people now call the digital revolution. Steven Levy \citeyearpar{levy84} has traced the origin of the word to the notorious Tech Model Railroad Club ---one of the very first groups of computer enthusiasts--- at the Massachusetts Institute of Technology. A \textit{hack} was defined by members of the TMRC as ``an article or project without constructive end'' and ``a project undertaken on bad self-advice''. 

This work is \emph{not} about vandals, ``cybercriminals'' or, as Clifford Stoll \citeyearpar{stoll89} chooses to call them, ``varmints''. Instead, it takes an interest in the newer generations of technologically-savvy do-it-yourselfers, heirs to a well-documented tradition of creating, tinkering, making and modifying machines, circuitry, computers and art-inspired artefacts as well as the spaces they have recently begun to occupy: grassroots, independent physical environments in which they meet and socialise, share information and build (sometimes) elegant machines and computer programmes, brought together by a common affinity towards science, technology, politics and the arts.

Partly due to the widespread stereotype of hackers as criminals, it is not widely known that theirs is a culture of institutional origin \citep{thomas02}, that stands atop strong philosophical foundations that date back to the late 1950s, influenced by several movements and currents before that. In 1984, Steven Levy coined the term ``Hacker Ethic'' to describe a pre-existing, yet unwritten compendium of the principles by which hackers abided. These principles revolved around notions of free information and knowledge, inherent apprehension towards authority and the formal establishment and a then-uncommon enchantment with science and technology. While the basic dogmas of the Hacker Ethic remain in essence to this day, scholars, intellectuals and younger hackers have re-interpreted and elaborated on them, accumulating in the process a vast conceptual and theoretical foundation from which this work benefits.

The advent of hackerspaces, however, seems to signal a turn back to the essence of the original Hacker Ethic. Most notably, a resurgent interest in hardware, re-vitalised in part by dropping costs of materials and widespread availability of instructional material online \citep{Kuznetsov}. Yet there seems to be more to the hackerspace phenomenon. 

The purpose of this work is thus to gain understanding into these relatively new environments and the people who conform them in the hopes of studying how their relationships, interactions and social exchanges contribute to facilitate learning and skills-building within their own habitats and further, to examine how such skills can ultimately lead to potentially significant discoveries and inventions. So far, governments and formal educational institutions seem mostly unaware or uninterested in hackerspaces. It is the intention of this work to shed light on what I hypothesise to be serious incubators for highly skilled and motivated individuals working on potentially ground-breaking innovations.

%art ant technology