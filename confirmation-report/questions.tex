\section{Research Questions}

This work revolves around a set of three questions. The first one poses a wider, more general intellectual puzzle, akin to what the French call \textit{une probl\'{e}matique}, providing a central theme throughout the whole of the dissertation. The second and third questions are deliberately devised to be more specific, to be answered in one or two core chapters.

\begin{list}{\labelitemi}{\leftmargin=0em}
\item[] \textbf{Primary question}. What is it that makes spatial proximity and co-presence so important in relation to hackerspaces? Why are hackers coming back together in a time where such proximity has been consistently portrayed as being no longer important?
\item[] \textbf{Secondary question}. How does physical contact shape hackers' behaviours, social relationships and group dynamics? In turn, how do these relationships and the everyday rituals that accompany them, influence their ability to learn, share and foster their highly technical and specialised skill-sets and knowledge within these admittedly informal environments?
\item[] \textbf{Secondary question}. Is there serious potential for ground-breaking innovation within hackerspaces? Can hackers' ventures have value (economic or otherwise) beyond their own social circles? If so, can they be considered as apt substitutes to more established centres for R\&D in developing countries, where such centres are scarce?
\end{list}