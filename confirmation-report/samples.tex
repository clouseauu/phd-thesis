\section{Writing samples}
\label{samples}

Please note that these two samples are taken out of a wider context. They should be understood not as whole logical units but rather as parts of two separate subchapters inside which they make more sense. The two subchapters from which these samples are taken are gladly available if needed.

\subsection{Sample 1: Hacker Origins and Philosophy}
\label{history}

The tradition of hacking has its origins within a formal academic environment. Indeed, the word ``hacker'' as a descriptor of the creative, technologi\-cal\-ly-savvy and computer enthusiasts first came into use at the Massachusetts Institute of Technology, where it derived from the older tradition of harmless and ingenious pranks some students devised (and still devise) at the University's campus \citep{levy84}\footnote{\textit{Hacking}, in this context, is a wonderful MIT tradition whereby students come up with elaborate jokes that demonstrate their technical prowess as well as their sense of humour. One example of a memorable hack was the placement of a police car on the top of the Institute's \textit{Great Dome}. For a full list of hacks, see \texttt{http://hacks.mit.edu/}.}.

As a result of the media's adoption of the term as a synonym for computer criminals, academic use of the word is heterogeneous at best and discordant at worst. Some scholars and researchers in the field of information technology and computer security have chosen to follow the media's trend, using the term to describe malicious computer users, thus making literature on the subculture that originated at MIT in the late 1950s extremely difficult to distinguish. In his book \textit{Hacker Culture}, author Douglas Thomas \citeyearpar{thomas02} takes a rather neutral stand on the etymological divergences of the word, choosing instead to focus on the consequences of such divergences: ``the very definition of the term `hacker' is widely and fiercely disputed by both critics and participants in the computer underground'', a fact that, in his view, ``gives a clue to both the significance and the mercurial nature of the subculture itself''. Such eclectic positions, however, add an additional layer of complexity to the task of performing a thorough literature review on \textit{any} of the word's conveyed meanings.

Indeed, the concept of hackers as a collective is as filled with almost as many contradictions as the word itself. While it would be easy to classify them into the formal side of Theodore Roszak's \citeyearpar{roszak69} radical cultural division of the post-war years, as members of a closed inner circle, or a techno-elite, I set out to prove that, in fact, hackers are quite the opposite, having more in common with Roszak's concept of counter-culture (which, as shall be discussed, is composed of many heterogeneous groups). Other scholars, such as McKenzie Wark \citeyearpar{wark04} have gone as far as comparing hackers to Marx's concept of the \textit{proletariat} ---presenting them as the exploited class of the information age.

Steven Levy's \textit{Hackers: Heroes of the Computer Revolution} \citeyearpar{levy84} is widely recognised as a fundamental, landmark piece of literature when it comes to historically documenting the emergence of the original hackers as an identifiable group and subculture. While not academic in nature, Levy's work can not only be considered an obligatory reference but also a document of historical significance itself, having provided an initial written declaration of philosophical principles, to which he dedicated an entire chapter of his book. Levy described what he called `The Hacker Ethic' ---a set of undeclared maxims that seemed to have originated along with the genesis of the movement, when the very first hackers lurked building 26 at MIT, hoping to harness unused, idle time from one of the first (and quite primitive) computers ever assembled:

% In the mid-’80s, following a rash of computer break-ins by teenagers with personal computers, true hackers stood by in horror as the general public began to equate the word — their word — with people who used computers not as instruments of innovation and creation but as tools of thievery and surveillance levy2010


\begin{quotation}
\ldots the dozen or so hackers were reluctant to acknowledge that their tiny society, on intimate terms with [the computer], had been slowly and implicitly piecing together a body of concepts, beliefs and mores.

The precepts of this revolutionary Hacker Ethic were not so much debated as silently agreed upon. No manifestos were issued. No missionaries tried to gather converts.
\end{quotation}

\noindent
If not the first, Levy was certainly amongst the earliest of theorists who saw the necessity to explicitly declare those ``concepts, beliefs and mores'', which he dubbed \textit{The Hacker Ethic}, consisting of six initial precepts:

\begin{enumerate}
\item Access to computers ---and anything which might teach you something about the way the world works--- should be unlimited and total. Always yield to the Hands-On Imperative!
\item All information should be free
\item Mistrust Authority - Promote Decentralization
\item Hackers should be judged by their hacking, not bogus criteria such as degrees, age, race or position
\item You can create art and beauty in a computer
\item Computers can change your life for the better
\end{enumerate}

\noindent
Inherent in such principles were ideas of freedom of information, anti-establishment and outright technological determinism, most of which remain unchanged to this day. Levy's precepts provide a clear, coherent basis to perform an analytical dissection of the emergence of the hacker movement. Precepts 1 and 2 summarise the movement's views on issues of freedom of information and their discrepancies with increasing restrictions on copyright and patent law, particularly for software but increasingly towards other types of ideas and works. Hackers were born within ---and still share close ties to--- an academic environment. Across the world, universities and academic institutions still foster and cherish the sharing of information. Gift economies are prevalent amongst academics, who continually build upon each other's work and who, just like those first hackers, are flattered ---not threatened--- to see others use and expand their own work.

Precepts 3 to 5 present an interesting perspective on Hackers' world-views, particularly when it comes to their relation with the wider spectrum of society. By making a personal interpretation of these three principles, I conclude that today's Hackers are heirs to the wider bohemian tradition that began in Europe during the 19th Century and subsequently evolved into the 1960s and 1970s counter-cultural movements, with which they shared similar views, tastes and political ideas.

Lastly, I judge the last precept to represent a growing trend towards technological determinism, reflective of the growing view that human and social development are closely tied to the advance of technology. 
%insert very very brief intro to tech determinism


\subsection{Sample 2: The Many Implications of Polanyi's Epistemology}
\label{polanyi}

The transmission of certain skills and knowledge, argued Michael Polanyi, requires a cooperative process of dialogue and unspoken understanding made possible by human contact through physical proximity, interaction and a certain degree of cultural affinity. Of course, this postulate is not new. Since Plato (and perhaps earlier), philosophers have identified a human dimension involved in learning. Polanyi's angle, however, is unique in a number of ways. Indeed, to fully apprehend it, one must examine these within a historical frame.

First, Polanyi dared to challenge a prevailing epistemological paradigm, (referred to as ``the objective ideal'') that became prevalent as a result of the ``scientific triumphs'' \citep[p.17]{gelwick77} that took place after the Copernican revolution but the causes of which began far earlier, with several historical developments that were, in his view, instrumental in leading to it.

Amongst them was the gradually increasing tension between two concepts he described as moral scepticism and moral perfectionism. From Greece onwards, the phenomenalist view that moral knowledge was unat\-tainable continued and grew in the works of modern thinkers and philosophers such as the Marquis de Sade, Nietzche and Rimbaud. Modern existentialism, Polanyi argued, uses moral scepticism to attack social morality as artificial and hypocritical. This progressive devaluation of moral values, however, could not by itself sway society on its way towards the mass adoption of the objective ideal: the degradation of judeo-christian morality caused partly by the advent of scientific rationalism led some to pursue such values even more passionately. Attacks on christianity during the Enlightenment led to the transposition of moralistic dogmas away from religion and ``into man's secular thoughts'' \citep[p.57]{polanyi66}, thus creating a strange collision between the rationality and objectivity sought by the nascent enlightened society with a secularised moral demand for social improvement. 

While Polanyi conceded that this tension initially led to the improvement of ``almost every human relationship, both private and public'', he also claimed that the increasing radicalisation of these views not only posed ``dangerous internal contradictions'' \citep[p.58]{polanyi66} but had led to the rise of existentialism, nihilism and further, Marxism, the tragic result being the discredit of ``all explicit expressions of morality''. The tension between the two sides, thus, resulted in a ``paradox of morality'' which ultimately led modern society to seek solace in absolute and almost thoughtless objectivity.

Second in the set of consequences leading to the wide adoption of the objective ideal (and the dawn of the ``modern scientific revolution'') was, according to Polanyi, the mechanisation of society. By means of a simple but eloquent account of the process leading to the growth of mechanism, he discussed this point with a brief historic recount, starting from Greek times, specifically Pythagoras, who came to view the world ``exclusively in terms of numbers''. Numerical relationships as beautiful as those found within triangles, thought Pythagoras, were present in all aspects of the world. Post Copernicus (and even more so after Kepler) the universe came to be seen as a mechanical system running with clock-work precision ---a fact that delighted those who studied it. \citet[p.7]{polanyi58} noted that Kepler regarded his own discoveries in ``ecstatic communion'', amazed and bewildered by the universal precision found in god's creation. Yet, after Galileo and Newton, things changed. Rather than being an inherent universal force, mathematics went on to be regarded simply as a summary of experience:

\begin{quote}
Numbers and geometrical forms are no longer assumed to be inherent as such in Nature. Theory no longer reveals perfection; it no longer contemplates the harmonies of Creation \ldots `pure' mathematics, formerly the key to nature's mysteries, became strictly separated from the \emph{application} of mathematics to the formulation of empirical laws. \citep[p.28]{polanyi58}
\end{quote}

What was to follow would be an almost inevitable progression from Descartes' rationalism to zealous objectivity marked the decline of the Enlightenment and the dawn of Mechanism, as proclaimed by many voices, the loudest of which was that of Ernst Mach, from the Vienna school of positivism and his book \textit{Die Mechanik}, published in 1883. 

It was to be Polanyi, then, who would restore confidence in human intuition by demonstrating an ever-present human element intrinsic in the act of knowing:

\begin{quote}
Both a craftsman and a novice are capable of identifying a particular tool and indicating some of its functions, but what distinguishes the craftsman's knowledge from that of the novice lies in the former's ability to use the tool subsidiarily in order to focus on the object of his craft. In the case of the scientist, he is not only identified by his explicit knowledge of given premises, but by his subsidiary utilization of those premises in the practice of science \citep[pp23-24]{kane84}.
\end{quote}

This analogy is particularly enlightening in the sense that it illustrates how Polanyi's thesis has extensive implications within today's prevailing systems for the acquisition of knowledge. Polanyi's unusual career pathway ---from the exact sciences to sociology and philosophy--- made him acutely aware of the need to ground his precepts on modern foundations, applicable to and compatible with modern practices and standard conventions of generally accepted scientific practices. Indeed, it was the scientific process itself that he sought to reform, for he reckoned that its very conception relied on principles which were ignored and sometimes expressly denied by many of his colleagues. 

Polanyi understood that no scientific theory can (nor should it) achieve perfect objectivity. As such, the boundaries of knowledge are ever broadened by reliance on certain presuppositions ---sets of which he dubbed ``subsidiary knowledge'', reached by using the senses as ``clues'': vehicles through which one can transcend into more abstract realms that ultimately lead to the acquisition of focal awareness or focal knowledge. Both types of awarenesses ---focal and subsidiary--- are necessary for new knowledge to occur.

Each new discovery (acquisition of explicit knowledge) is accompanied by a number of ``premises'' which do not (and cannot) stand on their own and that, until proven, remain as tacit or a-critical pieces of subsidiary awareness. Yet, they are essential to broadening the boundaries of scientific comprehension. Polanyi judged the almost obsessive scientific strive for objectivity to be misguided, arguing that all aspects of scientific discovery: data, theory and experimentation relied not only on said unproven subsidiary knowledge but also on the personal interpretation of these elements. Polanyi described these ``premises'' as by-products of proven theories ---by becoming proven themselves, they move into the realm of the explicit. If they are disproved, though, they are replaced, often generating Kuhn-style paradigm shifts.

In making such bold affirmations, however, Polanyi ran into an apparent dilemma, as, at first glance, it would sound like a recipe for the advocacy of blind faith and/or the lenient type of ``science'' practised by fanatics and supporters of pseudo-scientific belief systems. Indeed, insofar as personal interpretation (or subjectivity) is regarded as an integral part of the scientific method, all claims, including unsubstantiated or even fraudulent ones, would gain an aura of validity which would, very rapidly, over-flood the gates of serious inquiry to the point where it would ``run the risk of discontinuity and ultimately, inertia'' \citep[p.37]{kane84}. 

As a consummate and respected scientist, Polanyi was firmly committed to the rigorous experimental techniques of accepted scientific practices and intimately familiar with their minutiae, virtues and flaws. Consequently he was perfectly aware of this apparent predicament, which he addressed by means of the application of a concept he dubbed the ``conviviality of science'' whereby said gates are guarded by scientific communities, bound together by common thought, solid standards, deep tradition and a desire for accuracy and rectitude\footnote{One could replace these last two concepts with, simply, ``the truth''. However, keeping with Polanyi's canons an absolute concept such as truth does not do the point presented any favours.}. 

Ever the scientist, Polanyi sought to improve the act of knowing by correctly identifying its idiosyncrasies, a fact that marks the second particular aspect that differentiates his epistemological system from preceding ones also considering a human element, as it is grounded on present-day reasoning and meant to aid understanding and resolving the present-day problems associated with the acquisition of knowledge: ``man can transcend his own subjectivity by striving passionately to fulfil his personal obligations to universal standards'' \citep[p.4]{polanyi58}.

A third aspect differentiating Polanyi's theory from earlier ones is its advocacy for strong independence as a necessary condition for the production and transmission of knowledge. Polanyi recounts his conversations with a ``leading communist theoretician'' by the name of Bukharin\footnote{We assume that Polanyi refers to Nikolai Bukharin, a	 Soviet politician executed by Stalin in 1938.} \citep{polanyi66}, who argued that pure research ---\textit{scientia gratia scientis}--- was ``a morbid symptom of class society''. Polanyi was appalled by Bukharin's appraisal, which he saw as a reflection of what he considered to be an over-mechanised and utilitarian society, where express and useful purpose was required for all endeavours, including the pursuit of knowledge. Relegating research activities to central control, he argued, significantly decreased their effectiveness, as a certain degree of flexibility vastly improves the success of epistemological ventures. Communism had no room for the type of serendipitous discoveries that lead Pasteur to give birth to the field of immunology and Flemming to discover Penicillin.

Polanyi viewed state-controlled science as an early warning of the advent and growth of totalitarianism across Europe, pre World War II \citep{polanyi36}. By committing to central plans in the pursuit of knowledge, ``thinkers'' (scientists, scholars, intellectuals) relinquished their power to contradict the powerful by appealing to the truth. Judging by what was to become the immediate political state of Europe, Polanyi's arguments proved to be eerily accurate.
